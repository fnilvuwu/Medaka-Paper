Dalam penelitian ini, MobileNetV2 digunakan untuk klasifikasi objek berbasis gambar. Model ini merupakan versi penyempurnaan dari arsitektur MobileNetV1 sebelumnya. MobileNetV2 adalah bagian dari Convolutional Neural Network (CNN) yang dirancang khusus untuk perangkat dengan sumber daya terbatas seperti smartpohone, tablet, perangkat IoT, dan sistem tertanam. Dikembangkan oleh Google AI dan dipublikasikan pada tahun 2018, model ini menawarkan beberapa keunggulan yang telah disorot dalam berbagai studi, termasuk arsitekturnya yang ringan, kinerja yang kompetitif, serta kompatibilitasnya dengan berbagai kerangka kerja pembelajaran mendalam \cite{8578572,akay2021deep}. Model ini menggabungkan inverse residual block dengan linear bottleneck, yang meningkatkan efisiensi tanpa mengorbankan akurasi. Keunggulan efisiensi ini membuat MobileNetV2 cocok untuk aplikasi real-time pada perangkat berdaya rendah. 

%Struktur umum dari model MobileNetV2 dapat dilihat pada Gambar: \ref{fig: MobileNetV2 Architecture}. 
%\begin{figure}[h]
%\includegraphics[width=0.9\textwidth]
%{Images/MobileNetV2.png}
%\caption{General MobileNetV2 Architecture.
%\label{fig: MobileNetV2 Architecture}}
%\end{figure}

Dalam penelitian \cite{golcuk2023classification}, gambar varietas Cicer arietinum (kacang chickpea) dimasukkan ke dalam dua arsitektur. Pertama, menggunakan transfer learning dengan fine-tuning pada model CNN MobileNetV2 untuk klasifikasi. Kedua, arsitektur hybrid yang menggabungkan lapisan Long Short-Term Memory (LSTM) untuk mempertimbangkan fitur temporal. Hasilnya menunjukkan akurasi $92.3\%$ untuk model pertama dan $92.97\%$ untuk model hybrid, menunjukkan keberhasilan tinggi dalam mengklasifikasikan gambar varietas Cicer arietinum. Selain itu, penulis \cite{sen2021face} menerapkan pendekatan deep learning menggunakan MobileNetV2 dalam kerangka kerja PyTorch dan OpenCV Python untuk deteksi masker selama COVID-19. Model mereka secara efisien mengidentifikasi penggunaan masker yang benar, menunjukkan adaptabilitas MobileNetV2 untuk berbagai tugas berbasis gambar, mulai dari klasifikasi pertanian hingga aplikasi kesehatan masyarakat. Studi-studi ini menegaskan efektivitas MobileNetV2 dalam mencapai akurasi tinggi sambil mempertahankan efisiensi komputasi untuk penerapan di dunia nyata.


Dalam sebuah studi sebelumnya\cite{gulzar2023fruit}, penulis memperkenalkan Special Head yang terdiri dari lima lapisan berbeda ke dalam arsitektur MobileNetV2. Model lapisan klasifikasi asli digantikan dengan Special Head, menghasilkan versi modifikasi yang disebut TL-MobileNetV2. Adaptasi ini mencapai akurasi sebesar $99\%$ $—3\%$ lebih tinggi daripada MobileNetV2 standar, dengan tetap mempertahankan tingkat error yang sangat rendah, hanya $1\%$. Ketika dibandingkan dengan AlexNet, VGG16, InceptionV3, dan ResNet, TL-MobileNetV2 menunjukkan kinerja yang lebih unggul. Sementara itu, penulis \cite{indraswari2022melanoma} mengimplementasikan MobileNetV2 untuk klasifikasi kanker melanoma, melaporkan akurasi hingga $85\%$ pada dataset ISIC Archive, mengungguli model seperti ResNet50V2, InceptionV3, dan InceptionResNetV2. Studi lain \cite{shahoveisi2023application} mengevaluasi empat jaringan saraf konvolusional Xception, ResNet50, EfficientNetB4, dan MobileNet—untuk mendeteksi penyakit karat pada tiga tanaman komersial penting. Hasilnya menunjukkan EfficientNetB4 sebagai yang paling akurat (akurasi rata-rata = $94.29\%$), diikuti oleh ResNet50 ($93.52\%$). Meskipun MobileNetV2 sedikit tertinggal dalam perbandingan ini, efisiensi komputasinya membuatnya layak untuk aplikasi dengan sumber daya terbatas, memperkuat fleksibilitasnya di bidang medis dan pertanian. Penelitian lebih lanjut \cite{hamid2022smart} menggunakan MobileNetV2, sebuah deep learning convolutional neural network (DCNN) untuk klasifikasi biji. Sebanyak 14 kelas biji berbeda digunakan dalam eksperimen. Hasilnya menunjukkan akurasi sebesar $98\%$ dan $95\%$ pada set pelatihan dan pengujian, secara berturut-turut.

\begin{table}[H]
\caption{Gap description in building research motivation.\label{tab: tablegap}}
	\begin{adjustwidth}{-\extralength}{0cm}
		\newcolumntype{C}{>{\centering\arraybackslash}X}
		\begin{tabularx}{\fulllength}{m{5mm} m{8.5cm} m{8.5cm}}
			\toprule
            \textbf{No.}& \textbf{Previous Study}& \textbf{Current Study}\\
			\midrule
\multirow[m]{0.5}{*}{1} & 
Arsitektur hybrid menggabungkan lapisan Long Short-Term Memory (LSTM), yang juga memperhitungkan fitur data temporal dalam klasifikasi. \cite{golcuk2023classification} & Data gambar menampilkan empat warna latar belakang yang berbeda: merah, hitam, biru, dan hijau, dengan pencahayaan dari atas untuk memastikan ikan terlihat jelas.\\
                   \midrule
\multirow[m]{0.5}{*}{2}  & Implementasi pendekatan deep learning menggunakan framework MobileNetV2, yang terintegrasi dengan PyTorch dan OpenCV dalam Python, untuk deteksi masker selama COVID-19. \cite{sen2021face} & Kami mengembangkan aplikasi ini menggunakan Jupyter Notebook, TensorFlow, Roboflow, dan OpenCV untuk identifikasi \textit{Oryzias celebensis} dan \textit{Oryzias javanicus}.\\
                   \midrule
\multirow[m]{0.5}{*}{3} & Implementasi MobileNetV2 untuk Klasifikasi Kanker Melanoma. MobileNetV2 menunjukkan akurasi yang lebih tinggi dibandingkan dengan ResNet50V2, InceptionV3, dan InceptionResNetV2. \cite{indraswari2022melanoma} & Pengembangan arsitektur MobileNetV2 menggunakan pendekatan transfer learning dengan padding dataset dan lapisan tambahan untuk Classification Layer. Lapisan ini terdiri dari 5 komponen: sebuah Flatten layer dan dua Dense layer dengan fungsi aktivasi ReLU.\\
                  \midrule
\multirow[m]{0.5}{*}{4} & Dengan memanfaatkan MobileNetV2, sebuah jaringan saraf konvolusional pembelajaran mendalam (DCNN), untuk klasifikasi biji. \cite{hamid2022smart} & Kami mengembangkan versi modifikasi: P-MobileNetV2 (berbasis MobileNetV2) dan P-VGG16 (berbasis VGG16) untuk mengevaluasi efek dari perubahan arsitektur yang kami lakukan.\\                     \bottomrule
		\end{tabularx}
	\end{adjustwidth}
\end{table}

MobileNetV2 digunakan untuk perangkat mobile seperti smartphone dan tablet, yang berfokus pada efisiensi komputasi dan ukuran model yang lebih kecil, karena MobileNetV2 dirancang khusus untuk perangkat mobile dan aplikasi embedded vision. Seperti yang terlihat dalam eksperimen klasifikasi buah yang membandingkan kinerja antara MobileNetV2 dan Inceptionv3, hasilnya menunjukkan bahwa MobileNetV2 memiliki tingkat akurasi yang lebih baik, yang berarti performanya lebih unggul dibandingkan Inceptionv3. Dalam eksperimen kasus berbeda yang membandingkan kinerja MobileNetV2 dengan DenseNet121 untuk klasifikasi terumbu karang, hasilnya menunjukkan bahwa MobileNetV2 lebih optimal untuk perangkat dengan daya komputasi terbatas, serta lebih ringan dan cepat \cite{karnadi2024klasifikasi,utomo2025perbandingan}. Selain itu, arsitektur MobileNetV2 telah terbukti efektif dalam meningkatkan efisiensi komputasi tanpa mengorbankan akurasi, yang merupakan aspek penting untuk implementasi model ini pada perangkat mobile. Efisiensi komputasi yang tinggi memungkinkan implementasi di dunia nyata, di mana sumber daya komputasi mungkin terbatas \cite{maulana2024deteksi}.


Berkenaan dengan Tabel \ref{tab: tablegap}, dalam konteks ini, kami sedang mengembangkan sistem identifikasi dan klasifikasi untuk ikan \textit{O. celebensis} dan \textit{O. javanicus} menggunakan Deep Learning dan Transfer Learning yang diimplementasikan pada MobileNetV2 yang dimodifikasi. Kami percaya pendekatan inovatif kami akan memberikan dampak signifikan dalam proses identifikasi dan klasifikasi. Metodologi komprehensif kami menggabungkan teknik-teknik mutakhir, yang kami yakini dapat diadaptasi untuk berbagai tantangan identifikasi dan klasifikasi dalam aplikasi spesifik industri, khususnya sistem modern untuk mengidentifikasi spesies ikan endemik yang terancam punah.

