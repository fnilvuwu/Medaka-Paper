Data gambar ikan medaka yang berhasil diambil terdiri dari 661 gambar untuk \textit{O. celebensis} dan 886 gambar untuk \textit{O. javanicus}.

%\subsection{Captured Images}
\subsection{Image Pre-processing}

%\subsection{Rescaling Images}
Setelah mendapatkan dataset gambar, setiap gambar diberi padding untuk mencapai rasio aspek 1:1 dengan menambahkan piksel pada sisi yang lebih pendek menggunakan warna yang mirip dengan tepi gambar. Teknik ini jarang atau bahkan tidak pernah digunakan oleh peneliti lain, menjadikannya pendekatan yang unik dalam penelitian ini. Setelah menyesuaikan gambar ke rasio 1:1, gambar kemudian diubah ukurannya menjadi 224 x 224 piksel. Hasilnya dapat dilihat pada Gambar \ref{fig: padding}.



%%%%%%%%%%%%%%%%%%%%%%%%%%%%%%%%%%%%%%
\subsection{Experiment Results}
%%%%%%%%%%%%%%%%%%%%%%%%%%%%%%%%%%%%%%


Perbandingan kinerja setiap model pada dataset yang dipadding dan tidak dipadding disajikan dalam Tabel \ref{tab: perform}, yang menyoroti perbedaan akurasi dan efektivitas antara kedua pendekatan tersebut.


\begin{table}[H]
\caption{Accuracy of padding and non-padding images.\label{tab: perform}}
	\begin{adjustwidth}{-\extralength}{0cm}
		\newcolumntype{C}{>{\centering\arraybackslash}X}
		\begin{tabularx}{\fulllength}{CCCCCC}
			\toprule

\rowcolor{lightyellow}
\textbf{Image} & \textbf{Model} & \textbf{Sensitivity}& \textbf{Precision} & \textbf{F1 Score} & \textbf{Accuracy}\\
			\midrule
\multirow[m]{1}{*}{Non-padding}	& MobileNetV2& 98.2 & 93.8	& 96 & 96.95\\
                   \midrule
\multirow[m]{1}{*}{padding} & MobileNetV2	& 96.3	& 87.6	& 91.93& 93.9\\
                   \midrule
\multirow[m]{1}{*}{Non-padding} & P-MobileNetV2& 79& 98.4	& 87.6	& 89\\
                  \midrule
\multirow[m]{1}{*}{\textbf{padding}} & \textbf{P-MobileNetV2} & \textbf{98.46} & \textbf{98.46}	& \textbf{98.46}	& \textbf{98.78}\\
                   \midrule
\multirow[m]{1}{*}{Non-padding} & VGG16	& 95.3	& 93.5	& 94.5	& 95.7\\
                   \midrule
\multirow[m]{1}{*}{padding} & VGG16	& 97.2	& 96.2	& 81	& 87\\
                   \midrule
\multirow[m]{1}{*}{Non-padding} & P-VGG16	& 63.3	& 98.4	& 77	& 76\\
                   \midrule
\multirow[m]{1}{*}{padding} & P-VGG16	&92.7 &98.4	&96.3	&96.3\\   
   \bottomrule
		\end{tabularx}
	\end{adjustwidth}
\end{table}

P-MobileNetV2 mengacu pada model yang dilatih menggunakan dataset yang telah dipad (padded dataset), di mana gambar disesuaikan ke rasio aspek 1:1 untuk meningkatkan performa klasifikasi dan konsistensi selama pelatihan. Kami telah memperoleh hasil grafis yang membandingkan akurasi ROC antara MobileNetV2 dengan P-MobileNetV2 yang telah dimodifikasi, bersama dengan VGG16 dan P-VGG16 yang juga dimodifikasi. Perbandingan ini ditunjukkan dengan jelas dalam grafik berikut:

\begin{figure}[h]
\includegraphics[width=12 cm]{Images/Accuracy dan ROC Model MobileNetV2.png}
\caption{Diagram Accuracy dan ROC Model MobileNetV2.
\label{fig: diagram Accuracy dan ROC Model MobileNetV2}}
\end{figure}

Diagram Akurasi dan ROC dari model MobileNetV2 menggambarkan performanya dalam tugas klasifikasi. Grafik akurasi menunjukkan perkembangan pembelajaran model seiring waktu, sedangkan kurva ROC (Receiver Operating Characteristic) mengevaluasi kemampuannya dalam membedakan antar kelas, dengan menonjolkan sensitivitas dan spesifisitas pada berbagai ambang batas.


\begin{figure}[h]
\includegraphics[width=12 cm]{Images/Accuracy dan ROC Model P-MobileNetV2.png}
\caption{Diagram Accuracy dan ROC Model P-MobileNetV2.
\label{fig: diagram Accuracy dan ROC Model P-MobileNetV2}}
\end{figure}

%The "Accuracy and ROC Diagram of P-MobileNetV2 Model" illustrates the model's performance metrics that were modified, including classification accuracy and the Receiver Operating Characteristic (ROC) curve, evaluating its predictive capability.

Diagram 'Akurasi dan ROC dari Model P-MobileNetV2' menggambarkan metrik kinerja model yang telah dimodifikasi, termasuk akurasi klasifikasi dan kurva Receiver Operating Characteristic (ROC), untuk mengevaluasi kemampuan prediksinya. \textbf{Diagram Akurasi dan ROC Model P-MobileNetV2} menggambarkan metrik kinerja model yang telah dimodifikasi, termasuk \textbf{akurasi klasifikasi} dan \textbf{kurva Receiver Operating Characteristic (ROC)}, yang mengevaluasi kemampuan prediktifnya.

\begin{figure}[h]
\includegraphics[width=12 cm]{Images/Accuracy dan ROC Model VGG16.png}
\caption{Diagram Accuracy dan ROC Model VGG16.
\label{fig: diagram Accuracy dan ROC Model VGG16}}
\end{figure}

Diagram ini menampilkan akurasi dan kurva Receiver Operating Characteristic (ROC) dari model VGG16, yang menggambarkan performa klasifikasinya serta kemampuannya dalam membedakan antar kelas berdasarkan tingkat positif benar (true positive rate) dan tingkat positif salah (false positive rate) selama proses evaluasi.

\begin{figure}[h]
\includegraphics[width=12 cm]{Images/Accuracy dan ROC P-VGG16.png}
\caption{diagram Accuracy dan ROC Model P-VGG16.
\label{fig: diagram Accuracy dan ROC P-VGG16}}
\end{figure}

Diagram ini menampilkan akurasi dan kurva Receiver Operating Characteristic (ROC) dari model P-VGG16 yang telah dimodifikasi, menyoroti peningkatan performa klasifikasi serta kemampuan yang lebih baik dalam membedakan antar kelas berdasarkan tingkat positif benar (true positive rate) dan tingkat positif salah (false positive rate) selama fase evaluasi.


%%%%%%%%%%%%%%%%%%%%%%%%%%%%%%%
\subsection{Discussion}
%%%%%%%%%%%%%%%%%%%%%%%%%%%%%%%

Dalam penelitian ini, kami berhasil mengembangkan model arsitektur MobileNetV2 menggunakan transfer learning dan dataset padding untuk mengklasifikasikan \textit{O. celebensis} dan \textit{O. javanicus}. Kami menambahkan beberapa lapisan pada Classification Layer yang terdiri dari 5 lapisan: Flatten, dua lapisan Dense dengan fungsi aktivasi ReLU (masing-masing 1024 dan 512 neuron), lapisan Dropout dengan rate 0.2, serta lapisan Dense akhir dengan dua neuron menggunakan aktivasi SoftMax. Modifikasi ini memungkinkan kami memanfaatkan fitur yang telah dipelajari oleh MobileNetV2 dengan bobot ImageNet sekaligus mengadaptasi model untuk tugas klasifikasi spesifik kami.

Hasil penelitian menunjukkan bahwa pendekatan ini secara efektif meningkatkan kinerja klasifikasi. Lapisan dense tambahan berfungsi sebagai lapisan ekstraksi fitur dan klasifikasi, yang memetakan fitur-fitur yang diekstraksi oleh MobileNetV2 ke dalam dua kelas target yaitu \textit{O. celebensis} dan \textit{O. javanicus}. Lapisan Dropout (dengan rate 0.2) membantu mencegah overfitting, memastikan model tetap memiliki generalisasi yang baik pada data baru. Fungsi aktivasi SoftMax pada lapisan dense akhir memastikan output model dapat diinterpretasikan sebagai probabilitas kelas, sehingga menyederhanakan interpretasi hasil akhir. Struktur ini meningkatkan akurasi sekaligus mempertahankan ketangguhan, membuat model menjadi andal untuk membedakan kedua spesies ikan tersebut. Diskusi lebih lanjut mengungkapkan bahwa penggunaan MobileNetV2 sebagai model dasar menawarkan beberapa keunggulan:

\begin{itemize}
    \item \textbf{MobileNetV2 secara khusus dirancang untuk perangkat mobile}, menjadikannya ringan dan efisien untuk penerapan dalam aplikasi seluler atau lingkungan dengan sumber daya terbatas.
    \item \textbf{Dengan memanfaatkan dataset padding dan transfer learning}, kami memanfaatkan pengetahuan pra-latih MobileNetV2 dari dataset berskala besar, yang secara signifikan mempercepat dan menyederhanakan proses pelatihan model. Pendekatan ini memungkinkan model mencapai akurasi tinggi dengan sumber daya komputasi terbatas.
\end{itemize}
Selain itu, lapisan dense yang ditambahkan meningkatkan ekstraksi fitur dan klasifikasi, sementara lapisan dropout memastikan ketangguhan terhadap overfitting. Fungsi aktivasi SoftMax pada lapisan akhir memberikan keluaran probabilitas yang dapat diinterpretasikan untuk setiap kelas target (\textit{Oryzias celebensis} dan \textit{Oryzias javanicus}). Arsitektur lengkap model yang dikembangkan ditunjukkan pada Gambar \ref{fig: Modified-MobileNetV2-Architecture}.

\begin{figure}[h]
\includegraphics[width=8 cm]{Images/Modified-MobileNetV2-Architecture.jpg}\\
\caption{Modified MobileNetV2 Architecture.
\label{fig: Modified-MobileNetV2-Architecture}}
\end{figure}

Kami telah mengevaluasi kinerja arsitektur MobileNetV2 menggunakan lapisan tambahan yang dikonfigurasi seperti dijelaskan dalam tinjauan sistem, serta menerapkan padding pada dataset untuk mengklasifikasikan spesies ikan \textit{O. celebensis} dan \textit{O. javanicus} dengan dataset yang terbatas. Evaluasi dilakukan menggunakan pengujian berbasis matriks kebingungan (confusion matrix) dan kurva ROC-AUC, dengan memanfaatkan dua jenis dataset: dataset tanpa padding dan dataset dengan padding. Dua arsitektur model digunakan untuk perbandingan, yaitu MobileNetV2 dan VGG16. Hasilnya menunjukkan bahwa MobileNetV2 berkinerja lebih baik pada dataset dengan padding dibandingkan kinerjanya pada dataset tanpa padding, serta lebih unggul dibandingkan VGG16 pada kedua jenis dataset tersebut.
