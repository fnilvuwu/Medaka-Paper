\textit{Oryzias celebensis} dan \textit{Oryzias javanicus} adalah dua spesies ikan endemik yang memiliki peranan penting dalam ekosistem perairan di Indonesia, khususnya di wilayah Sulawesi dan Jawa \cite{parenti2008phylogenetic,herder2022more}. Kedua spesies ini memiliki peran penting dalam menjaga keragaman hayati dan keseimbangan ekosistem \cite{nurdin2023perubahan}. Identifikasi spesies ikan langka merupakan langkah awal yang krusial dalam upaya konservasi. Dengan mengenali karakteristik morfologi, analisis genetika, dan pemahaman perilaku, kita dapat memahami lebih baik tentang spesies-spesies ini. Identifikasi juga membantu kita mengambil tindakan yang lebih efektif untuk melindungi habitat dan mencegah kepunahan.


\textbf{Genus Oryzias} termasuk dalam keluarga \textbf{Adrianichthyidae}. Ikan-ikan ini tersebar luas di wilayah Asia Selatan, Asia Timur, dan Asia Tenggara. Habitat alaminya meliputi persawahan, kolam, parit, dan danau \cite{kottelat2013fishes,magtoon2009revised}. \textit{Oryzias celebensis} dan \textit{Oryzias javanicus} merupakan dua spesies yang terancam punah di Indonesia. Ikan Medaka Sulawesi \textit{Oryzias celebensis} merupakan spesies endemik Pulau Sulawesi \cite{parenti2011endemism,eragradhini2020ekobiologi}. Sayangnya, Ikan Medaka merupakan ikan endemik yang terancam punah dan telah dinyatakan secara resmi oleh International Union for Conservation of Nature and Natural Resources (IUCN) \cite{serdiati2021perspectives}. Upaya konservasi sangat penting untuk mencegah kepunahan spesies ini. \textit{O. javanicus}: Ikan ini juga dikenal dengan nama Java Medaka. Habitatnya meliputi perairan tawar hingga payau. \textit{O javanicus} tersebar di sebagian besar wilayah Asia, termasuk Indonesia. Kehadirannya di alam liar semakin terancam akibat aktivitas perikanan yang tidak seimbang dengan upaya konservasi dan polusi air yang meningkat. Kedua spesies ini memiliki adaptasi yang unik terhadap lingkungan perairan mereka, dan peran mereka sebagai indikator sensitif terhadap kualitas air telah menjadikannya penting dalam pemantauan dan pengelolaan sumber daya air\cite{hasan2022checklist}.


%Genus Oryzias termasuk dalam famili Adrianichthyidae. Ikan-ikan ini tersebar luas di Asia Selatan, Asia Timur, dan Asia Tenggara. Habitat alaminya meliputi sawah, kolam, parit, dan danau . \textit{O. celebensis} dan \textit{O. javanicus} adalah dua spesies yang sangat terancam punah di Indonesia. Celebes Medaka (O. celebensis) merupakan spesies endemik di Pulau Sulawesi . Sayangnya, ikan Medaka adalah spesies endemik yang terancam punah dan telah secara resmi dinyatakan demikian oleh International Union for Conservation of Nature (IUCN). Upaya konservasi sangat penting untuk mencegah kepunahan spesies-spesies ini. Oryzias javanicus: Ikan ini juga dikenal sebagai Medaka Jawa. \cite{lamba2023habitat}


Sayangnya, kita menyadari bahwa populasi \textit{O. Celebensis} dan \textit{O. Javanicus} semakin menurun, membawa kedua spesies ini ke ambang kepunahan. Kondisi ini menjadi lebih mengkhawatirkan karena keduanya termasuk dalam kategori ikan langka yang secara endemik ditemukan di Indonesia, khususnya di wilayah Sulawesi dan Jawa. Ancaman terhadap habitat alami mereka, perubahan iklim, dan aktivitas manusia yang merusak lingkungan perairan semakin memperparah keadaan mereka \cite{MOKODONGAN2015150}. Dalam upaya untuk melindungi dan melestarikan spesies-spesies langka ini, para peneliti sering menghadapi dilema etis dalam proses identifikasi. Metode konvensional yang biasa digunakan, seperti penggunaan ikan yang sudah mati atau menempatkan ikan di luar air yang berpotensi merugikan, tidak lagi dapat diterima dalam konteks keberlanjutan dan konservasi lingkungan \cite{voss2014assessing}.


Untuk mengatasi tantangan ini, inovasi dalam bidang kecerdasan buatan, khususnya teknologi deep learning, menawarkan solusi yang menjanjikan. Dengan memanfaatkan kemampuan komputasi yang canggih, para peneliti dapat menggunakan model deep learning untuk melakukan klasifikasi dan identifikasi spesies ikan, termasuk \textit{Oryzias celebensis} dan \textit{Oryzias javanicus}, secara cepat dan efisien. Pendekatan ini tidak hanya lebih ramah lingkungan, tetapi juga memungkinkan pemantauan yang lebih efektif terhadap populasi ikan langka ini, membantu upaya konservasi untuk menjaga keberlangsungan hidup mereka di masa depan.
