% \subsection{Sumber Data}
% Dataset yang digunakan dalam penelitian ini terdiri dari gambar primer dan sekunder ikan Medaka. Data primer diperoleh dengan memotret langsung \textit{Oryzias javanicus} dan \textit{Oryzias celebensis} di dalam akuarium menggunakan kamera digital. Data sekunder dikumpulkan dari sumber terbuka, termasuk situs penelitian, forum akuatik, dan basis data akses terbuka. Setelah kurasi, dataset akhir berjumlah 792 gambar yang mencakup berbagai kondisi pencahayaan dan sudut pandang untuk meningkatkan ketahanan (robustness) dan generalisasi. Seluruh dataset yang telah dianotasi serta kode yang digunakan dalam penelitian ini akan tersedia secara publik dalam sebuah repositori daring setelah publikasi. Karena hanya dilakukan fotografi non-invasif di akuarium, persetujuan etik untuk eksperimen hewan tidak diperlukan.

% \subsection{Praproses Data}
% Seluruh gambar distandarkan dalam orientasi, resolusi, dan keseimbangan warna. Restorasi citra (peningkatan kontras, penghapusan artefak), augmentasi data (pembalikan, rotasi $90^{\circ}$, $180^{\circ}$, dan $270^{\circ}$), serta normalisasi diterapkan. Anotasi dilakukan menggunakan platform Roboflow, dengan bounding box dan label kelas disimpan dalam format YOLO. Dataset dibagi menjadi set pelatihan, validasi, dan pengujian dengan proporsi masing-masing 80%, 10%, dan 10%.

% \subsection{Pelatihan Model}
% Arsitektur YOLOv8 disesuaikan (fine-tuning) menggunakan dataset yang telah dianotasi. Untuk mengevaluasi ketahanan model, digunakan validasi silang 5-lipatan (5-fold cross-validation). Dalam pembelajaran ensemble, beberapa model YOLOv8 dengan inisialisasi bobot berbeda digabungkan menggunakan Weighted Box Fusion (WBF). Pelatihan dan evaluasi dilakukan pada GPU NVIDIA dengan hiperparameter default YOLOv8 kecuali dinyatakan lain. Implementasi dijalankan menggunakan Python dengan kerangka kerja YOLOv8 dari Ultralytics.

% \subsection{Metrik Evaluasi}
% Kinerja model dievaluasi menggunakan metrik COCO, termasuk mean Average Precision (mAP@0.5:0.95), presisi, recall, dan kesalahan regresi bounding box (RMSE). Matriks kebingungan (confusion matrix) dan kurva loss juga dianalisis untuk menilai kinerja klasifikasi dan konvergensi model.

% \subsection{Penggunaan Kecerdasan Buatan Generatif}
% Kecerdasan buatan generatif (ChatGPT, OpenAI) hanya digunakan untuk meningkatkan kejelasan dan bahasa dalam naskah. Tidak ada alat AI yang digunakan untuk menghasilkan data, melakukan analisis, atau memengaruhi interpretasi hasil.

\subsection{Sumber Data}
Dataset yang digunakan dalam penelitian ini terdiri atas data citra primer dan sekunder dari ikan Medaka. Data primer diperoleh melalui pemotretan langsung di akuarium terhadap spesies \textit{Oryzias javanicus} dan \textit{Oryzias celebensis} menggunakan kamera digital. Data sekunder dikumpulkan dari sumber daring yang dapat diakses publik, termasuk situs penelitian, forum komunitas akuatik, dan basis data terbuka. Setelah proses kurasi, dataset akhir berjumlah 792 citra yang mencakup variasi kondisi pencahayaan dan sudut pandang, sehingga meningkatkan ketahanan (robustness) dan kemampuan generalisasi. Seluruh data anotasi dan kode akan tersedia secara publik dalam repositori daring setelah publikasi. Karena pengumpulan data hanya dilakukan melalui pemotretan non-invasif di akuarium, persetujuan etik tidak diperlukan.

\subsection{Praproses Data}
Untuk memastikan konsistensi, seluruh citra distandarkan berdasarkan orientasi, resolusi, dan keseimbangan warna sebelum digunakan dalam pelatihan. Proses praproses mencakup normalisasi data serta penerapan strategi augmentasi untuk meningkatkan ketahanan dan mengurangi risiko overfitting (kecocokan berlebih). Augmentasi meliputi penyesuaian properti warna seperti rona, saturasi, dan kecerahan guna mensimulasikan variasi kondisi pencahayaan dan lingkungan; translasi serta penskalaan untuk merepresentasikan objek pada posisi dan jarak yang berbeda; serta pembalikan horizontal guna memperkaya keragaman dataset. Teknik mosaik juga diterapkan dengan menggabungkan empat citra ke dalam satu sampel pelatihan sehingga model terekspos pada komposisi adegan yang lebih kompleks dan interaksi antarobjek. Selain itu, penghapusan acak pada sebagian area citra dilakukan untuk mendorong model mengenali fitur yang kurang menonjol tetapi tetap relevan. Anotasi dilaksanakan menggunakan platform Roboflow, dengan kotak pembatas (bounding box) dan label kelas disimpan dalam format YOLO. Dataset kemudian dibagi menjadi subset pelatihan (80\%), validasi (10\%), dan pengujian (10\%).

\subsection{Pelatihan Model}
Arsitektur YOLOv8 disesuaikan (fine-tuned) menggunakan dataset yang telah dianotasi. Untuk mengevaluasi ketahanan model, digunakan metode 5-fold cross-validation. Dalam pembelajaran ensemble, beberapa model YOLOv8 dengan bobot awal berbeda digabungkan menggunakan Weighted Box Fusion (WBF), yaitu metode yang mengintegrasikan prediksi berdasarkan tumpang tindih kotak pembatas dan skor kepercayaan. Proses pelatihan dan evaluasi dilakukan pada lingkungan GPU NVIDIA, dengan mengikuti hiperparameter default dari kerangka kerja YOLOv8 kecuali dinyatakan lain. Implementasi dijalankan menggunakan Python dengan pustaka Ultralytics YOLOv8.

\subsection{Metrik Evaluasi}
Kinerja model dievaluasi menggunakan metrik COCO, termasuk mean Average Precision (mAP@0.5:0.95), precision, recall, dan galat regresi kotak pembatas (Root Mean Square Error / RMSE). Analisis tambahan, seperti matriks kebingungan (confusion matrix) dan kurva kehilangan pelatihan (loss curves), juga digunakan untuk mengevaluasi lebih lanjut performa klasifikasi dan konvergensi model.
