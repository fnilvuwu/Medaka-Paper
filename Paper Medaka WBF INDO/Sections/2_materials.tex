\subsection{Sumber Data}
Dataset yang digunakan dalam penelitian ini terdiri dari gambar primer dan sekunder ikan Medaka. Data primer diperoleh dengan memotret langsung \textit{Oryzias javanicus} dan \textit{Oryzias celebensis} di dalam akuarium menggunakan kamera digital. Data sekunder dikumpulkan dari sumber terbuka, termasuk situs penelitian, forum akuatik, dan basis data akses terbuka. Setelah kurasi, dataset akhir berjumlah 792 gambar yang mencakup berbagai kondisi pencahayaan dan sudut pandang untuk meningkatkan ketahanan (robustness) dan generalisasi. Seluruh dataset yang telah dianotasi serta kode yang digunakan dalam penelitian ini akan tersedia secara publik dalam sebuah repositori daring setelah publikasi. Karena hanya dilakukan fotografi non-invasif di akuarium, persetujuan etik untuk eksperimen hewan tidak diperlukan.

\subsection{Praproses Data}
Seluruh gambar distandarkan dalam orientasi, resolusi, dan keseimbangan warna. Restorasi citra (peningkatan kontras, penghapusan artefak), augmentasi data (pembalikan, rotasi $90^{\circ}$, $180^{\circ}$, dan $270^{\circ}$), serta normalisasi diterapkan. Anotasi dilakukan menggunakan platform Roboflow, dengan bounding box dan label kelas disimpan dalam format YOLO. Dataset dibagi menjadi set pelatihan, validasi, dan pengujian dengan proporsi masing-masing 80%, 10%, dan 10%.

\subsection{Pelatihan Model}
Arsitektur YOLOv8 disesuaikan (fine-tuning) menggunakan dataset yang telah dianotasi. Untuk mengevaluasi ketahanan model, digunakan validasi silang 5-lipatan (5-fold cross-validation). Dalam pembelajaran ensemble, beberapa model YOLOv8 dengan inisialisasi bobot berbeda digabungkan menggunakan Weighted Box Fusion (WBF). Pelatihan dan evaluasi dilakukan pada GPU NVIDIA dengan hiperparameter default YOLOv8 kecuali dinyatakan lain. Implementasi dijalankan menggunakan Python dengan kerangka kerja YOLOv8 dari Ultralytics.

\subsection{Metrik Evaluasi}
Kinerja model dievaluasi menggunakan metrik COCO, termasuk mean Average Precision (mAP@0.5:0.95), presisi, recall, dan kesalahan regresi bounding box (RMSE). Matriks kebingungan (confusion matrix) dan kurva loss juga dianalisis untuk menilai kinerja klasifikasi dan konvergensi model.

\subsection{Penggunaan Kecerdasan Buatan Generatif}
Kecerdasan buatan generatif (ChatGPT, OpenAI) hanya digunakan untuk meningkatkan kejelasan dan bahasa dalam naskah. Tidak ada alat AI yang digunakan untuk menghasilkan data, melakukan analisis, atau memengaruhi interpretasi hasil.