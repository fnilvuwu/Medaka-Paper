\subsection{Metode Ensemble AdaBoost}
Salah satu kontribusi utama dari penelitian ini adalah penerapan metode ensemble AdaBoost pada model YOLOv8. Algoritme AdaBoost mengurangi keterbatasan detektor individual dengan memberikan bobot lebih tinggi pada sampel yang salah klasifikasi pada iterasi berikutnya. Reweighting iteratif ini terbukti sangat bermanfaat untuk citra Medaka yang menantang, seperti ikan berukuran kecil, sebagian tubuh yang terhalang, atau kondisi pencahayaan yang bervariasi.

Dibandingkan dengan studi sebelumnya yang hanya mengandalkan detektor model tunggal untuk pengenalan ikan, pembelajaran ensemble telah terbukti secara konsisten meningkatkan ketahanan pada tugas deteksi objek \cite{refX,refY}. Temuan kami sejalan dengan tren ini, menunjukkan bahwa ensemble YOLOv8–AdaBoost mencapai nilai mAP dan recall yang lebih tinggi dibandingkan model individual mana pun, yang mengindikasikan bahwa strategi boosting dapat berperan penting dalam mengatasi ketidakseimbangan kelas dan deteksi objek kecil di lingkungan akuatik.

\subsection{Validasi Silang 5-Lipatan untuk Generalisasi Model}
Kontribusi utama lainnya dari penelitian ini adalah implementasi validasi silang 5-lipatan, yang memastikan evaluasi model yang kuat dan bebas bias. Dengan memutar peran validasi pada lima subset, kami meminimalkan risiko overfitting terhadap partisi data tertentu. Kinerja stabil di seluruh lipatan menyoroti kemampuan model untuk melakukan generalisasi dengan baik, bahkan di bawah variasi kompleksitas latar belakang, posisi ikan, dan kualitas citra.

Sejalan dengan penelitian terdahulu dalam visi komputer ekologi, validasi silang telah diakui sebagai teknik penting untuk memvalidasi model ketika dataset relatif kecil atau heterogen \cite{refZ}. Hasil kami memperluas bukti ini pada deteksi ikan Medaka, mengonfirmasi bahwa generalisasi yang andal dapat dicapai meskipun ukuran dataset relatif terbatas.

\subsection{Implikasi untuk Konservasi dan Pemantauan}
Integrasi AdaBoost dan validasi silang ke dalam sistem berbasis YOLOv8 untuk deteksi spesies yang terancam punah memiliki implikasi penting bagi pemantauan konservasi. Identifikasi akurat terhadap spesies ikan kecil yang secara visual mirip memungkinkan penilaian yang lebih andal terkait dinamika populasi, kualitas habitat, dan keanekaragaman hayati. Hal ini sangat relevan untuk genus \textit{Oryzias}, yang mencakup spesies dengan status konservasi mengkhawatirkan di Asia Tenggara.

Dengan mengurangi kesalahan deteksi pada individu yang langka atau terhalang, pendekatan kami mendukung protokol pemantauan ekologi di mana keandalan lebih diprioritaskan daripada kecepatan. Sistem ensemble ini sangat sesuai untuk analisis offline pada data lapangan, melengkapi strategi konservasi yang ada seperti survei populasi, pemetaan habitat, dan pemantauan keanekaragaman hayati jangka panjang.

\subsection{Arah Penelitian Masa Depan}
Meskipun metode ensemble meningkatkan akurasi, metode ini menimbulkan biaya komputasi yang lebih tinggi, sehingga penerapan waktu nyata menjadi kurang layak. Penelitian masa depan perlu mengeksplorasi teknik ensemble yang lebih ringan, model pruning, atau knowledge distillation untuk menyeimbangkan akurasi dan efisiensi. Selain itu, perluasan dataset dengan menambahkan lebih banyak spesies Medaka dan kondisi lingkungan yang beragam akan meningkatkan skalabilitas dan transferabilitas sistem.

Arah penelitian menjanjikan lainnya adalah integrasi informasi temporal dari video, alih-alih memperlakukan frame secara independen. Koherensi temporal dapat semakin mengurangi kesalahan negatif pada ikan yang bergerak. Akhirnya, studi mendatang dapat membandingkan boosting ensemble dengan strategi agregasi lain, seperti bagging atau weighted box fusion, untuk mengidentifikasi solusi optimal dalam deteksi spesies akuatik.

%%%%%%%%%%%%%%%%%%%%%%%%%%%%%%%%%%%%%%%%%%
\section{Kesimpulan}

Penelitian ini menunjukkan efektivitas integrasi AdaBoost dan validasi silang 5-lipatan dengan model YOLOv8 untuk mendeteksi dan mengklasifikasikan spesies ikan Medaka yang langka. Peningkatan ini secara signifikan memperbaiki presisi, recall, dan mAP, serta mampu mengatasi tantangan seperti deteksi objek kecil dan lingkungan bawah air yang kompleks secara visual. Dengan memperkuat ketahanan dan mengurangi overfitting, pendekatan yang diusulkan berkontribusi pada pemantauan spesies terancam yang lebih andal, mendukung inisiatif konservasi yang bergantung pada data ekologi yang akurat.

Namun, peningkatan akurasi diperoleh dengan mengorbankan efisiensi komputasi, sehingga membatasi penerapan waktu nyata. Penelitian masa depan sebaiknya berfokus pada strategi ensemble ringan, algoritme boosting alternatif seperti Gradient Boosting atau XGBoost, serta pendekatan deep learning hibrida yang menyeimbangkan akurasi dengan kecepatan. Perluasan dataset pada spesies \textit{Oryzias} tambahan dan integrasi informasi temporal dari video juga dapat lebih meningkatkan kemampuan generalisasi model. Pada akhirnya, penelitian ini menetapkan dasar metodologis untuk pengembangan sistem deteksi objek yang teroptimasi dan berorientasi konservasi di lingkungan akuatik.