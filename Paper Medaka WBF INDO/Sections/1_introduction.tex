Kemajuan kecerdasan buatan (Artificial Intelligence/AI) telah memberikan kontribusi signifikan terhadap upaya konservasi satwa liar, khususnya dalam mengotomatisasi identifikasi dan pemantauan spesies. Teknologi ini sangat berharga di lingkungan yang sulit dilakukan observasi manual, seperti ekosistem bawah air, di mana faktor-faktor seperti kekeruhan dan pencahayaan rendah memperumit visibilitas \citep{LeCun2015, Liu2019}. Sistem berbasis AI mampu memproses sejumlah besar data visual dari gambar maupun video untuk membantu mendeteksi keberadaan spesies, melacak perilakunya, serta mendukung strategi konservasi dengan intervensi manusia yang minimal \citep{Kalafi2018}.

Dalam konteks ini, model deteksi objek semakin populer karena kecepatan dan efisiensinya pada aplikasi waktu nyata. Arsitektur You Only Look Once (YOLO), khususnya versi terbarunya yaitu YOLOv8, menyempurnakan iterasi sebelumnya dengan peningkatan akurasi dan efisiensi arsitektural, sehingga cocok untuk tugas yang menantang seperti pengenalan spesies akuatik. YOLOv8 secara efektif mengatasi permasalahan degradasi citra yang umum terjadi di bawah air, seperti kekeruhan, pencahayaan yang kurang baik, dan obstruksi \citep{Redmon2017}.

Penelitian kami secara khusus menargetkan deteksi dua spesies ikan Medaka yang penting—\textit{Oryzias javanicus} dan \textit{Oryzias celebensis}. Kedua spesies ini menghadirkan tantangan pemantauan tersendiri karena populasi yang semakin menurun dan kompleksitas pengambilan gambar di bawah air. Untuk memperkuat penelitian ini, kami membangun dataset khusus yang terdiri dari gambar yang dikumpulkan secara manual menggunakan kamera serta gambar yang tersedia secara publik dari sumber internet. Dataset ini berisi total 2.016 gambar, yang terbagi menjadi 1.857 gambar (92%) untuk pelatihan dan 159 gambar (8%) untuk pengujian. Kami menggunakan Roboflow untuk anotasi manual dataset, serta melakukan praproses dan augmentasi yang diperlukan \citep{Dang2020}.

Langkah praproses meliputi penyesuaian orientasi otomatis, pengubahan ukuran gambar menjadi standar 640×640 piksel, dan penyaringan anotasi kosong. Untuk meningkatkan ketahanan model dan memperluas generalisasi, augmentasi seperti pembalikan horizontal dan vertikal, serta rotasi (90°, 180°, dan 270°) diterapkan \citep{Arbogast2016}. Pilihan metodologis ini bertujuan mengurangi risiko overfitting dan mensimulasikan berbagai kondisi dunia nyata yang mungkin dihadapi model.

Untuk tugas deteksi, kami menggunakan pendekatan berbasis YOLOv8 dengan validasi silang 5-lipatan (5-fold cross-validation) guna melatih lima model individu. Prediksi dari masing-masing model kemudian disempurnakan dan digabungkan menggunakan Weighted Box Fusion (WBF), sebuah metode ensemble tingkat lanjut yang meningkatkan kotak deteksi akhir dengan mengevaluasi skor kepercayaan dan tumpang tindih—pada akhirnya memperkuat kinerja deteksi model secara keseluruhan \citep{Solovyev2021, Leow2015}. Temuan kami menunjukkan bahwa pendekatan ensemble ini memberikan peningkatan signifikan pada nilai rata-rata presisi (mean Average Precision/mAP) dibandingkan dengan model YOLOv8 tunggal. Perlu dicatat bahwa meskipun metode ensemble meningkatkan waktu inferensi, metode ini secara signifikan memperkuat ketahanan model, terutama dalam mendeteksi target kecil atau kurang terlihat seperti ikan Medaka. Melalui penelitian ini, kami menetapkan acuan dasar bagi pengembangan metodologi ensemble yang ringan dan dapat diterapkan pada deteksi objek bawah air dengan memanfaatkan model deteksi waktu nyata yang canggih \citep{Salimi2016}.