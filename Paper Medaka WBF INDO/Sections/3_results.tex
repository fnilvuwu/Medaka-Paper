Bagian ini memberikan deskripsi ringkas mengenai hasil eksperimen, interpretasinya, serta kesimpulan utama yang dapat ditarik. Eksperimen disusun dalam tiga tahap: (i) fine-tuning pada satu model YOLOv8, (ii) validasi silang 5-lipatan untuk menilai kemampuan generalisasi, dan (iii) ensemble berbasis AdaBoost dari model-model dengan kinerja terbaik. Evaluasi kinerja dilakukan menggunakan metrik COCO (mAP, presisi, recall) dan matriks kebingungan (confusion matrix).

\subsection{Eksperimen 1: Fine-Tuning YOLOv8}

Eksperimen pertama melatih YOLOv8 selama 100 epoch dengan ukuran batch 16 pada dataset campuran (70% pelatihan, 20% validasi, 10% pengujian). Hasil yang diperoleh sebagai berikut:

\begin{itemize}
\item Presisi stabil pada $\sim$0,80;
\item Recall mencapai 0,90, menunjukkan deteksi yang andal untuk kedua spesies;
\item Matriks kebingungan menunjukkan performa tinggi pada \textit{O. celebensis} (presisi 0,96, recall 0,95), namun performa sedikit lebih rendah pada \textit{O. javanicus} (presisi 0,87, recall 0,81).
\end{itemize}

Secara keseluruhan, model tunggal ini berkinerja baik, namun cenderung kurang mendeteksi \textit{O. javanicus} pada kondisi yang menantang.

\subsection{Eksperimen 2: Validasi Silang 5-Lipatan}

Untuk mengevaluasi generalisasi, dataset dibagi menjadi lima lipatan (80% pelatihan, 20% validasi). Setiap subset berperan sebagai validasi sekali, sementara empat subset lainnya digunakan untuk pelatihan.

Hasil utama:
\begin{itemize}
\item Rata-rata mAP@0.5 pada seluruh lipatan: 0,78;
\item Rata-rata presisi: 0,77; rata-rata recall: 0,82;
\item Model 4 menunjukkan performa paling stabil di seluruh metrik.
\end{itemize}

Hal ini mengonfirmasi bahwa YOLOv8 mampu melakukan generalisasi dengan baik pada citra ikan Medaka yang belum pernah dilihat, sekaligus mengurangi risiko overfitting.

\subsection{Eksperimen 3: Ensemble dengan AdaBoost}

Sebuah ensemble dibangun menggunakan lima model hasil validasi silang dengan pembobotan AdaBoost. Model dengan kesalahan lebih tinggi diberi bobot lebih besar pada iterasi berikutnya.

Ensemble menunjukkan peningkatan yang jelas:
\begin{itemize}
\item mAP@0.5 meningkat menjadi 0,81 (dari 0,78 pada validasi silang);
\item mAP@0.5:0.95 meningkat menjadi 0,63;
\item Presisi naik menjadi 0,82; recall menjadi 0,86.
\end{itemize}

Ensemble ini terbukti sangat efektif dalam mengurangi kesalahan klasifikasi pada ikan yang berukuran kecil atau terhalang.

\subsection{Gambar, Tabel, dan Skema}

Komposisi dataset dirangkum pada Tabel~\ref{tab:data_sources}, dan contoh gambar yang dianotasi ditampilkan pada Gambar~\ref{fig:dataset_hybrid}. Dinamika pelatihan model (kurva loss, grafik presisi-recall) disajikan pada Gambar~\ref{fig:exp1_curve}–\ref{fig:exp2_error_class_loss}, sedangkan kinerja ensemble dirangkum pada Gambar~\ref{fig:ensemble_results}.

\begin{table}[H]
\caption{Distribusi dataset gambar ikan Medaka.}
\label{tab:data_sources}
\begin{tabularx}{\textwidth}{lCCC}
\toprule
\textbf{Spesies} & \textbf{Data Primer} & \textbf{Data Sekunder} & \textbf{Total} \\
\midrule
\textit{O. javanicus} & 257 & 178 & 435 \\
\textit{O. celebensis} & 287 & 70  & 357 \\
\midrule
\textbf{Total} & \textbf{544} & \textbf{248} & \textbf{792} \\
\bottomrule
\end{tabularx}
\end{table}

\begin{figure}[H]
\centering
\includegraphics[width=0.8\textwidth]{Images/dataset_hybrid.jpg}
\caption{Contoh gambar Medaka yang dianotasi dari dataset campuran.}
\label{fig:dataset_hybrid}
\end{figure}