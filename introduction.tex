\section{Introduction}
Technological advancements had made possible for automation of fish detection. This allows for the conservation of endangered animals can be initiated earlier as the initial stage using the artificial intelligence (AI) technology. The fundamental initial stage is the detection and classification of endangered species through the application of deep learning algorithms as an AI model. The application of AI enables the use of camera traps or advanced sensors to automatically identify endangered species, obviating the necessity for direct human involvement in their habitat. Furthermore, AI is capable of analyzing copious amounts of data from a multitude of sources, including satellite images and sound recordings, in order to discern patterns in animal behavior or distribution that are challenging to discern manually. Furthermore, AI can predict environmental threats to specific animal populations, such as climate change or deforestation, thereby enabling more rapid and accurate conservation efforts. Additionally, AI enables the simulation of the optimal environment for a particular species, thus facilitating more effective conservation planning in the future. AI can even be used in conjunction with robots to automatically manage conservation areas. By leveraging this technology, endangered animal conservation efforts can become more efficient and sustainable.

% Paragraf kedua bercerita tentang deep learning, tentang deteksi dan klasifikasi berbasis data citra dapat dikembangkan untuk hewan menggunakan YOLOv8.
In the realm of deep learning, image-based detection and classification methods, such as YOLOv8, can be developed for animal conservation. YOLOv8 is an evolution of the SSD (Single Shot MultiBox Detector), a two-stage object detection model. This algorithm is optimized for real-time object detection and consistently achieves high-precision results across various applications.

% Paragraf ketiga bercerita tentang beberapa penelitian yang terkait dengan deteksi dan klasifikasi ikan menggunakan YOLOv8.
Several recent studies have utilized YOLOv8 for fish detection and classification, showing the great potential of this technology in aquatic ecosystem conservation. One study successfully used YOLOv8 to identify various fish species in coral reefs with high accuracy, even in low-light conditions or blurred images due to water particles. Another study utilized YOLOv8 to monitor the movement and population of fish in rivers and lakes, where real-time analysis results enabled the detection of endangered species or invasion of non-native species. The speed and efficiency of YOLOv8 in processing underwater image data is an advantage, as the aquatic environment often does not support direct visual observation by humans. In addition, some projects have integrated YOLOv8 with underwater autonomous vehicles (UAVs) or automated underwater cameras, which can work continuously to monitor marine ecosystems without interruption. The results of fish classification using YOLOv8 also make important contributions to the study of fish behavior, migration patterns, and ecosystem interactions, which are important for conservation policies. The use of YOLOv8 in this fish study shows that AI technology is not only relevant for land, but can also play a key role in maintaining the balance of aquatic ecosystems. These studies are clear examples of how advanced AI algorithms can be applied to understand, protect and manage animal populations in aquatic habitats.


% Paragraf keempat bercerita tentang kekurangan dan kelemahan YOLOv8 terkait deteksi dan klasifikasi ikan langka yang berukuran kecil. Bercerita tentang ide dan teori dalam mengatasi kekurangan dan kelemahan tersebut.
While YOLOv8 has many advantages in fish detection and classification, there are some drawbacks that arise when applied to small, rare fish species. YOLOv8, like other object detection algorithms, faces challenges in identifying small objects in the image, especially if the fish is moving fast or among visual complexities such as coral reefs or water particles. YOLOv8's accuracy can also degrade when small rare fish occupy only a fraction of the image pixels, making it difficult for the algorithm to recognize specific details. To overcome this drawback, one idea is to increase the resolution of the input image or utilize super-resolution techniques, which can magnify and clarify the details of small fish before they are processed by the detection algorithm. In addition, the theory of using multi-scale training and anchor-free detection can be applied, where YOLOv8 is trained with different sizes of objects to accommodate the scale variation of fish in real environments. The addition of more specific training data for small fish as well as the use of augmented data can help the model become more adaptive to environmental variations. The integration of other technologies, such as acoustic sensors that complement visual detection, can also be a multidisciplinary approach to improve the accuracy of small fish species classification. By continuously developing this solution, YOLOv8 can be improved to be more effective in detecting and protecting rare small fish species in aquatic ecosystems. 

Many studies have focused on modifying the underlying architecture of YOLOv8 to enhance model accuracy such as YOLO-SAG, which introduces changes to the backbone and neck to improve performance. Another common approach involves modifying activation layers to boost accuracy and efficiency. For instance, integrating architectures and recently developed activation functions, e.g., Softplus, has shown promise in improving model performance. However, these approaches are primarily concerned with architectural modifications as the core of the model.

Despite these advancements, there remains a notable gap in research focused on statistical approaches to enhancing YOLOv8's performance without altering its architecture. As new boosting techniques continue to emerge, such as LPBoost, TotalBoost, BrownBoost, XGBoost, MadaBoost, and LogitBoost, the application of these statistical methods to YOLOv8 remains underexplored. Investigating these techniques within the context of object detection models such as YOLOv8 could provide valuable insights and further improve accuracy without the need for complex architectural modifications.

Hence this paper is made to fill the gap with the use of a mixture dataset, i.e., a collection of data compoundsl primary and secondary data. The use of the mixed datasets is necessary since primary data for rare or endangered animals is very difficult to capture in sufficient numbers. Therefore, secondary data from sources such as the internet is used to supplement the dataset. Once the dataset was collected, preprocessing was performed, including object segmentation, to ensure consistency of the data in terms of resolution, size, brightness, and other parameters, resulting in uniform and fair input. The main focus of this research is not on how the datasets are collected, but on how artificial intelligence (AI) technology can be implemented for the conservation of rare and endangered animals. Thus, this research focuses on the utilization of AI in facing conservation challenges, especially through automatic detection and classification that supports conservation efforts more efficiently and effectively.

The data used in this research utilizes a mixture dataset, which is a combined dataset between primary and secondary data. This is done because primary data for rare or endangered animals is very difficult to obtain. So that secondary data is needed as additional data obtained from the Internet. This hybrid data is then preprocessed with object segmentation so that the input data is uniform in terms of resolution, size, brightness and others of equal value (fair). The substance of this paper is not on how the dataset is obtained, but how the implementation of AI can be used for nature conservation issues, especially rare and endangered animals can be done using AI technology.

%%% Kontribusi dan Kebaruan %%%
This paper successfully found a detection and classification model for Medaka fish. (\textit{Oryzias javanicus} and \textit{Oryzias celebensis}) The model used is based on YOLOv8 with a focus on setting up balanced and equitable data with a limited number of sample images with more than 90\% accuracy (as the first contribution of this paper). The model used is based on YOLOv8 with a focus on balanced and fair data preparation. Although the data sources in this experiment came from direct observation data (primary data) and data derived from various sources on the Internet (secondary data), the resulting model performed very well. A careful and targeted data acquisition framework is the second contribution of this paper. Data preparation or data pre-processing works with three stages, namely, the business process understanding stage that directs the right data sources, the data understanding stage that produces data specifications according to the required data, and the data preparation or cleaning understanding stage. With these stages, the model built produces high accuracy because the data sources have even and quality specifications. The next contribution is that the model is developed through three experiments, namely, parameter fine tuning techniques, 5-fold cross validation, and ensemble techniques using AdaBoost. The three experiments show that the AdaBoost ensemble method is slightly superior to the 5-fold cross validation method and the parameter fine tuning method. The final contribution is the proposal of several new performance measures including model risk and confidence levels that are developed from accuracy measures. 

The key contributions of this paper are as follows:
\begin{itemize}
    \item A dataset consists of two classes: the endangered fish species Oryzias javanicus and Oryzias celebensis.
    \item Application of 5-fold cross validation on YOLOv8 object detection model
    \item Implementation of ensemble techniques using AdaBoost on YOLOv8 object detection model
\end{itemize}