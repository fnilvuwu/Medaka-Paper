% \documentclass{elsarticle}
% %\documentclass[3p,times]{elsarticle}

% \usepackage{graphicx} % Required for inserting images
% \usepackage{booktabs}

\documentclass[10pt]{article}

\usepackage{fullpage}
\usepackage{setspace}
\usepackage{parskip}
\usepackage{titlesec}
\usepackage[section]{placeins}
\usepackage{xcolor}
\usepackage{breakcites}
\usepackage{lineno}
\usepackage{hyphenat}
\usepackage{multirow}
\usepackage{graphicx}
\usepackage{tabularx}

\PassOptionsToPackage{hyphens}{url}
\usepackage[colorlinks = true,
            linkcolor = blue,
            urlcolor  = blue,
            citecolor = blue,
            anchorcolor = blue]{hyperref}
\usepackage{etoolbox}
\makeatletter
% \patchcmd\@combinedblfloats{\box\@outputbox}{\unvbox\@outputbox}{}{%
%   \errmessage{\noexpand\@combinedblfloats could not be patched}%
% }%
\makeatother


\usepackage{natbib}


\renewenvironment{abstract}
  {{\bfseries\noindent{\abstractname}\par\nobreak}\footnotesize}
  {\bigskip}

\titlespacing{\section}{0pt}{*3}{*1}
\titlespacing{\subsection}{0pt}{*2}{*0.5}
\titlespacing{\subsubsection}{0pt}{*1.5}{0pt}


\usepackage{authblk}


\usepackage{graphicx}
\usepackage[space]{grffile}
\usepackage{latexsym}
\usepackage{textcomp}
\usepackage{longtable}
\usepackage{tabulary}
\usepackage{booktabs,array,multirow}
\usepackage{amsfonts,amsmath,amssymb}
\providecommand\citet{\cite}
\providecommand\citep{\cite}
\providecommand\citealt{\cite}
% You can conditionalize code for latexml or normal latex using this.
\newif\iflatexml\latexmlfalse
\providecommand{\tightlist}{\setlength{\itemsep}{0pt}\setlength{\parskip}{0pt}}%

\AtBeginDocument{\DeclareGraphicsExtensions{.pdf,.PDF,.eps,.EPS,.png,.PNG,.tif,.TIF,.jpg,.JPG,.jpeg,.JPEG}}

\usepackage[utf8]{inputenc}
\usepackage[english]{babel}


\usepackage{float}

\usepackage{enumitem}

\begin{document}

\title{\bf YOLOv8-WBF: Ensemble Learning for Reliable Detection of Endangered Medaka (Oryzias)}


\author[a,b]{Armin Lawi}
\author[a]{Muhammad Haerul}
\author[a]{Iman Mustika Ismail}
\author[a]{Rahmatullah R.}
\author[c]{Irma Andriani}
\author[d]{Andi Iqbal Baharuddin}

\affil[a]{Information Systems Study Program, Hasanuddin University, Makassar, Indonesia}
\affil[b]{B.J. Habibie Institute of Technology, Parepare, Indonesia}
\affil[c]{Department of Biology, Hasanuddin University, Makassar, Indonesia}
\affil[d]{Department of Fishery, Hasanuddin University, Makassar, Indonesia}


\vspace{-1em}


% \date{September 01, 2024}


\begingroup
\let\center\flushleft
\let\endcenter\endflushleft
\maketitle
\endgroup


\selectlanguage{english}
\begin{abstract}
Reliable detection of Medaka (Oryzias) fish is crucial for ecological monitoring and conservation, particularly for assessing population trends of endangered species. In this study, we evaluate the performance of a state-of-the-art deep learning model (YOLOv8) and an ensemble approach using Weighted Box Fusion (WBF) on a manually annotated dataset of Medaka images collected from online sources. Models were trained and validated using 5-fold cross-validation, and performance was assessed using COCO metrics, including mean Average Precision (mAP), precision, recall, and bounding box regression error. The YOLOv8-WBF ensemble achieved a mAP@0.5:0.95 of 0.578, representing an 8.04\% improvement over the best single model (0.535). It also reduced bounding box localization error (RMSE: 45.41 to 29.93) and improved classification loss, indicating more reliable detection of small and visually challenging fish instances. These accuracy gains came at the expense of computational efficiency, with inference requiring approximately five times more operations than a single YOLOv8 model. While less suited for real-time deployment, the ensemble approach provides more accurate population monitoring in offline ecological workflows, where reliability is often prioritized over speed. By reducing missed detections of small or occluded fish, our findings contribute to more robust biodiversity monitoring and establish a baseline for developing optimized ensemble and lightweight detection models in aquatic conservation domains.
\end{abstract}%

\sloppy


\maketitle

Kemajuan kecerdasan buatan (Artificial Intelligence/AI) telah memberikan kontribusi signifikan terhadap upaya konservasi satwa liar, khususnya dalam mengotomatisasi identifikasi dan pemantauan spesies. Teknologi ini sangat berharga di lingkungan yang sulit dilakukan observasi manual, seperti ekosistem bawah air, di mana faktor-faktor seperti kekeruhan dan pencahayaan rendah memperumit visibilitas \citep{LeCun2015, Liu2019}. Sistem berbasis AI mampu memproses sejumlah besar data visual dari gambar maupun video untuk membantu mendeteksi keberadaan spesies, melacak perilakunya, serta mendukung strategi konservasi dengan intervensi manusia yang minimal \citep{Kalafi2018}.

Dalam konteks ini, model deteksi objek semakin populer karena kecepatan dan efisiensinya pada aplikasi waktu nyata. Arsitektur You Only Look Once (YOLO), khususnya versi terbarunya yaitu YOLOv8, menyempurnakan iterasi sebelumnya dengan peningkatan akurasi dan efisiensi arsitektural, sehingga cocok untuk tugas yang menantang seperti pengenalan spesies akuatik. YOLOv8 secara efektif mengatasi permasalahan degradasi citra yang umum terjadi di bawah air, seperti kekeruhan, pencahayaan yang kurang baik, dan obstruksi \citep{Redmon2017}.

Penelitian kami secara khusus menargetkan deteksi dua spesies ikan Medaka yang penting—\textit{Oryzias javanicus} dan \textit{Oryzias celebensis}. Kedua spesies ini menghadirkan tantangan pemantauan tersendiri karena populasi yang semakin menurun dan kompleksitas pengambilan gambar di bawah air. Untuk memperkuat penelitian ini, kami membangun dataset khusus yang terdiri dari gambar yang dikumpulkan secara manual menggunakan kamera serta gambar yang tersedia secara publik dari sumber internet. Dataset ini berisi total 2.016 gambar, yang terbagi menjadi 1.857 gambar (92%) untuk pelatihan dan 159 gambar (8%) untuk pengujian. Kami menggunakan Roboflow untuk anotasi manual dataset, serta melakukan praproses dan augmentasi yang diperlukan \citep{Dang2020}.

Langkah praproses meliputi penyesuaian orientasi otomatis, pengubahan ukuran gambar menjadi standar 640×640 piksel, dan penyaringan anotasi kosong. Untuk meningkatkan ketahanan model dan memperluas generalisasi, augmentasi seperti pembalikan horizontal dan vertikal, serta rotasi (90°, 180°, dan 270°) diterapkan \citep{Arbogast2016}. Pilihan metodologis ini bertujuan mengurangi risiko overfitting dan mensimulasikan berbagai kondisi dunia nyata yang mungkin dihadapi model.

Untuk tugas deteksi, kami menggunakan pendekatan berbasis YOLOv8 dengan validasi silang 5-lipatan (5-fold cross-validation) guna melatih lima model individu. Prediksi dari masing-masing model kemudian disempurnakan dan digabungkan menggunakan Weighted Box Fusion (WBF), sebuah metode ensemble tingkat lanjut yang meningkatkan kotak deteksi akhir dengan mengevaluasi skor kepercayaan dan tumpang tindih—pada akhirnya memperkuat kinerja deteksi model secara keseluruhan \citep{Solovyev2021, Leow2015}. Temuan kami menunjukkan bahwa pendekatan ensemble ini memberikan peningkatan signifikan pada nilai rata-rata presisi (mean Average Precision/mAP) dibandingkan dengan model YOLOv8 tunggal. Perlu dicatat bahwa meskipun metode ensemble meningkatkan waktu inferensi, metode ini secara signifikan memperkuat ketahanan model, terutama dalam mendeteksi target kecil atau kurang terlihat seperti ikan Medaka. Melalui penelitian ini, kami menetapkan acuan dasar bagi pengembangan metodologi ensemble yang ringan dan dapat diterapkan pada deteksi objek bawah air dengan memanfaatkan model deteksi waktu nyata yang canggih \citep{Salimi2016}.

\subsection{Data Sources}
The dataset used in this study consisted of both primary and secondary images of Medaka fish. Primary data were obtained by directly photographing \textit{Oryzias javanicus} and \textit{Oryzias celebensis} in an aquarium using a digital camera. Secondary data were collected from publicly available sources, including research websites, aquatic forums, and open-access databases. The final dataset contained 792 images after curation, covering a variety of lighting conditions and viewing angles to enhance robustness and generalization. All annotated datasets and code used in this study will be made publicly available in an online repository upon publication. Since only non-invasive aquarium photography was conducted, no ethical approval was required for animal experimentation. 

\subsection{Data Preprocessing}
All images were standardized in orientation, resolution, and color balance. Image restoration (contrast enhancement, artifact removal), data augmentation (flipping, rotations of $90^{\circ}$, $180^{\circ}$, and $270^{\circ}$), and normalization were applied. Annotation was performed using the Roboflow platform, with bounding boxes and class labels stored in YOLO format. The dataset was split into training, validation, and test sets with proportions of 80\%, 10\%, and 10\%, respectively.

\subsection{Model Training}
We fine-tuned the YOLOv8 architecture on the annotated dataset. To evaluate robustness, 5-fold cross-validation was employed. For ensemble learning, multiple YOLOv8 models with different weight initializations were combined using Weighted Box Fusion (WBF). Training and evaluation were performed on an NVIDIA GPU with default YOLOv8 hyperparameters unless otherwise specified. The implementation was carried out in Python using the Ultralytics YOLOv8 framework.

\subsection{Evaluation Metrics}
Model performance was assessed using COCO metrics, including mean Average Precision (mAP@0.5:0.95), precision, recall, and bounding box regression error (RMSE). Confusion matrices and loss curves were also analyzed to assess classification performance and model convergence.

\subsection{Use of Generative AI}
Generative artificial intelligence (ChatGPT, OpenAI) was used solely to improve the clarity and language of the manuscript. No AI tools were used to generate data, perform analyses, or influence interpretation of results.

\section{Experiments}
    \subsection{Experimental environment and parameter configuration}
    % TODO
    The experimental setup for this study utilized an NVIDIA GeForce
    RTX4080 GPU with 16GB of memory, Python 3.8.18, PyTorch 1.7.1
    framework, and CUDA 12.0. The training is performed using a homemade wildlife dataset, and the experimental parameters are shown in
    Table 2.

    \subsection{Evaluation index}
    % TODO
    (1) Precision evaluation metrics: Precision (P), Recall (R), mean
    Average Precision (mAP), and F1 score. mAP@0.5 and mAP@0.5–0.95
    denote the mAP value at an IoU threshold of 0.5 and the average mAP
    value as the IoU ranges from 0.5 to 0.95 with a step size of 0.05,
    respectively. The formulas for each index are shown in Eqs. 12–15,
    where TP represents the number of true positive predictions, FP denotes
    the number of false positive predictions, and FN signifies the number of
    false negative predictions. (2) Speed evaluation metrics: inference time
    (time from preprocessed image input into the model to model output
    result), post-processing time, floating-point operations(GFLOPs), model
    size, and parameter.
    P = TP
    TP + FP (12)
    R = TP
    TP + FN
    
    \subsection{Experiment 1: Fine-Tuning YOLOv8}
    The first experiment was conducted by applying fine-tuning to the YOLOv8 model. YOLOv8 is one of the latest object detection models designed to detect and classify objects in real-time. In this experiment, the pre-trained YOLOv8 model was fine-tuned to the collected Medaka fish dataset. Fine-tuning involves adjusting the pre-trained weights using the Medaka fish data to allow the model to recognize the unique features of the species. \textit{Oryzias javanicus} and \textit{Oryzias celebensis}.
    
    The fine-tuning process is performed using a dataset that has been prepared through preprocessing and augmentation stages, with optimal hyperparameter settings. The goal of this experiment is to improve the accuracy of fish species detection and classification by using the knowledge gained from large datasets and adapting it to more specific datasets.
    
    Fine-tuning the YOLOv8 model is a crucial step to adapt the model's performance to the specific characteristics of the dataset used in this study. YOLOv8, as one of the most advanced object detection algorithms today, requires a fine-tuning process to optimize its ability to detect rare or endangered species. In this process, the pre-trained weights of the model were adjusted using training data from the mixture dataset we collected, which is a combination of primary and secondary data.
    
    The fine-tuning process allows the model to learn unique features of our dataset, such as specific environmental variations, lighting conditions, as well as special traits present in the target species. By applying an approach where some network layers remain frozen while other layers are updated, we were able to utilize the general knowledge gained from the large-scale dataset while still tailoring the model to the unique characteristics of this dataset. This strategy is important for improving detection accuracy, reducing the risk of overfitting, and ensuring the model is reliable in various environmental conservation scenarios.
    
    In this study, the YOLOv8 model is fine-tuned using a mixture dataset that has been divided into 80\% for training data, 20\% for testing data. The model was trained with optimally tuned hyperparameters, namely for 100 epochs, using a batch size of 16, with an automatic optimizer, and a learning rate of 0.01. The model training lasted for approximately 2 hours, utilizing GPU-based computing infrastructure to speed up the process.
    
    \subsection{Experiment 2: 5-Fold Cross Validation}
    In the second experiment, the model was evaluated using the 5-fold cross validation technique. Cross-validation is a method used to ensure that the model has stable performance and can be generalized to various subsets of data. In 5-fold cross-validation, the train data is divided into five subsets, where at each iteration, four subsets are used to train the model, and one subset is used for validation. This process is repeated five times so that each subset serves as a validation set once.
    
    The results of each iteration are then averaged to obtain more accurate performance metrics, such as precision, recall, and mean average precision (mAP). By using cross-validation, the risk of overfitting the model can be minimized, allowing the model to perform optimally on unseen data.
    
    Cross-validation is used to evaluate the reliability and generalizability of YOLOv8 model performance. In this study, we applied 5-fold cross-validation, where the dataset is divided into five subsets (folds). At each iteration, the model is trained using four subsets and tested on the remaining subset. This process is repeated five times, with each fold acting as a one-time validation set. Performance metrics, such as precision, recall, and mean average precision (mAP), are calculated at each iteration and then averaged to provide an accurate and reliable estimate of the model's overall capability.
    
    Cross-validation is important to reduce the risk of overfitting, because by dividing the data into multiple subsets and testing the model iteratively, we can ensure that the model performs consistently even if the data is divided in different ways. It also helps in ensuring that the model is more resilient to variations in the dataset and reliable when applied to data outside the training sample.
    
    In this research, the dataset is divided into 80\% (508 images) for the training set and 20\% (158 images) for the test set. The training set was further divided into five subsets (A, B, C, D, and E), each containing 127 images. The model was trained using four subsets as training set and one subset as validation set at each iteration. This training process was repeated five times, rotating among the five subsets. Illustration of experiment 2 or cross validation as shown in {Figure 2}.

    \begin{figure*}[ht]
        \centering
        \includegraphics[width=100mm]{image/Diagram5FoldCrossValidationMethodBlocks.png} 
        \vspace{-6mm}
        \caption{5-Fold Cross Validation}
        \label{fig:foldCrossValidation}
    \end{figure*}

    The model was trained using optimally tuned hyperparameters for 100 epochs, with batch size 16, auto optimizer, and learning rate 0.01. Each training iteration took about 2 hours, so the total time required to complete five cross-validations was approximately 10 hours.
    
    \subsection{Experiment 3: Ensemble Method (ADABoost)}
    The third experiment involved applying ensemble methods to further improve detection and classification performance. The method used is AdaBoost, an ensemble algorithm that works by combining multiple models to improve prediction accuracy. In this approach, the YOLOv8 model is combined with several other models, or variations of the YOLOv8 model with different hyperparameter settings, to form an ensemble system.
    
    The main motivation for using AdaBoost is to improve model performance by correcting the weaknesses of individual models that may fail to recognize certain patterns. The AdaBoost algorithm gives greater weight to the prediction errors of the previous model, so that the next iteration of the model focuses more on correcting those errors. In this way, the resulting model is more robust and accurate, and has a better ability to generalize predictions, even on complex or variable data.
    
    Ensemble learning is a technique used to improve model performance by combining predictions from multiple models. In this study, we apply an ensemble approach by combining multiple YOLOv8 models, where each model is trained with slightly different hyperparameters or trained on different subsets of data. The purpose of this ensemble is to reduce the variance and bias that may exist in each individual model, resulting in more accurate and reliable predictions. This ensemble approach can be done by averaging outputs (such as bounding box coordinates and confidence scores) or using a majority voting mechanism for classification. This technique helps improve overall robustness and predictive ability, especially when dealing with real-world data complexity and diversity.
    
    Experiments using the ADABoost ensemble technique were conducted by utilizing 5 models from the cross validation results in experiment 2. The ADABoost algorithm is as follows

    \begin{table*}[ht]
    \centering
        \begin{tabular}{l}
        \hline \vspace{-1mm} \\
        \vspace{1mm} \textbf{Algorithm: AdaBoost} \vspace{1mm} \\
        \hline \\
            \begin{tabular}[c]{@{}l@{}}
                1. \textbf{Input}: Dataset $D = \{(x_1, y_1), (x_2, y_2), \ldots ,(x_n, y_n)\}$,\\ 
                \qquad \qquad \quad Learner $\Gamma$ and the number of learning iteration $T$\\
                \vspace{-2mm} \\
                2. Initialize weight sample $w_i = \frac{1}{N}, \forall i = 1, 2, \ldots, N$ \\ \vspace{-2mm} \\
              
                3. Iterate, \textbf{for} $t=1 \textbf{ to } T$ \textbf{do} \\ \vspace{-2mm} \\
                \qquad (a) Train a weak learner $h_t$ from $D_t (\in D)$ to train sample $w_i$. %\\
                $h_t = \Gamma \left(D, D_t \right)$ \\ \vspace{-2mm} \\
                % \qquad \qquad \qquad \qquad $h_t = \Gamma \left(D, D_t \right)$ \\ \vspace{-2mm} \\               
                \qquad (b) Compute error of $h_t$:
                %\qquad \qquad \qquad \qquad 
                $\varepsilon_t = \frac{\sum_{i=1}^N w_i \cdot I \left( h_t \left( x_i \right) \neq y_i \right)}{\sum_{i=1}^N w_i}$ \\ \vspace{-1mm} \\
                
                \qquad (c) Compute the weight of $h_t$:
                %\qquad \qquad \qquad \qquad 
                $\alpha_t = \frac{1}{2}\ln\left( \frac{1-\varepsilon_t}{\varepsilon_t} \right)$ \\ \vspace{-2mm} \\
                
                \qquad (d) Assign: $w_i \leftarrow w_i \cdot e^{\left( \alpha_t \cdot I \left( h_t \left( x_i \right) \neq y_i \right)\right)}$ \\ \vspace{-2mm} \\
                
                4. \textbf{Output:} $H(x) = \text{sign} \sum_{t=1}^T \left(\alpha_t \cdot h_t (x)\right)  $ \\
                \vspace{1mm}
            \end{tabular} \\ 
        \hline
        \end{tabular}
    \end{table*}

    \begin{table*}[ht]
        \centering
        \caption{table}
        \vspace{3mm}
        \begin{tabular}{ccc}
            \toprule
            \textbf{Fine-Tuning} & \multicolumn{1}{c}{\textbf{Cross Validation}} & \multicolumn{1}{c}{\textbf{Ensemble (AdaBoost)}} \\ \midrule
            \includegraphics[width=45mm]{image/fine_tuning.png} & 
            \includegraphics[width=45mm]{image/cross_validation.png} & 
            gambar  \\
            \bottomrule
        \end{tabular}
    \end{table*}


\section{Results}
    
    \subsection{Mixture Data Preparation}
    This subsection presents the results of the mixed data preparation process with the stages described in the subsection~\ref{Sec:olahData}. 

Data preparation sequence:
\begin{enumerate}
    \item download digital image file 
    \item Restoration of all digital objects in the file by removing the noise in the file. For video files or moving images, restoration is performed on each frame of the captured image per-second.
    \item The file restoration process is performed by removing noise applying a Gaussian kernel function.
    \item All image files are then normalized to 640x640 pixels.
    \item Normalized files containing multiple objects are annotated according to their respective species names.
\end{enumerate}
    
    \subsubsection{Data Acquisition and Labeling}

    The data preparation stage was initially performed by collecting primary data from captured still images and moving videos into a main folder\footnote{Digital still image and moving video of an object hereafter are called digital object}. Next, all digital objects in a file were provided with bounding boxes and annotated with labels corresponding to their species. (The annotation results of the digital objects were stored as metadata along with other information such as location, time, authorship, etc., in a file). A summary of the dataset obtained using the methods described in subsection \ref{dataSources} can be seen in Table~\ref{table:data_sources} 

    \begin{table*}[ht]
        \centering
        \caption{Recapitulation of Primary and Secondary Data}
        \label{table:data_sources}
        \begin{tabular}{@{}clllc@{}}
            \toprule
            No. & \multicolumn{1}{c}{Fish Class} & \multicolumn{1}{c}{Data Primer} & \multicolumn{1}{c}{Data Sekunder} & Jumlah \\ \midrule
            1. & \textit{O. Javanicus} & \multicolumn{1}{c}{257} & \multicolumn{1}{c}{178} & \multicolumn{1}{c}{435} \\
            2. & \textit{O. Celebensis} & \multicolumn{1}{c}{287} & \multicolumn{1}{c}{70} & \multicolumn{1}{c}{357} \\ \midrule
            \multicolumn{1}{l}{} & \multicolumn{1}{c}{\textbf{Total}} & \multicolumn{1}{c}{\textbf{544}} & \multicolumn{1}{c}{\textbf{248}} & \multicolumn{1}{c}{\textbf{792}} \\ 
            \bottomrule
        \end{tabular}
    \end{table*}
    
    The mixture data consisted of 792 total images, i.e., 435 (55\%) images of O. Javanicus and 357 (45\%) O. celebensis fish images which after preprocessing were divided into subsets for each experiment. Primary data yielded 544 (69\%) fish and secondary data 248 (31\%) fish.

    \subsubsection{Image Restoration}
    Remove noise by applying brightness, contrast and gaussian kernel function settings. Size 

    
    \subsubsection{Normality Test of Data}
    A normality test is performed to show that the primary and secondary data in the dataset are equivalent.
    
    \begin{figure*}[ht]
    \centering
    \includegraphics[width=160mm, height=160mm,scale=0.5]{image/dataset_hybrid.jpg}
    \caption{Resulted mixture images data with their annotation respectively}
    \label{fig:dataset_hybrid}
    \end{figure*}
    
    \subsubsection{Data Splitting}
    Experiment 1: Data Train 634 (80\%), Data Test 158 (20\%) - Model Experiment 1
    
    Experiment 2: Train data is divided into 5 groups with an arrangement of 127 (16\%) each. Each group is named dataset A, B, C, D, and E. Then the model is cross validated with a combination of train data and validation data ratio of 4:1 or 80\% training data and 20\% validation data from 634 train data in experiment 1. Overall, train data 508 (64\%), Validation Data 127 (16\%), and Test Data 158 (20\%). The simulated cross validation experiment design can be organized as follows. \\
    Model 1: ABCD as Train, E as Validation (80,20) \\
    Model 2: ABCE as Train, D as Validation \\
    Model 3: ABDE as Train, C as Validation \\
    Model 4: ACDE as Train, B as Validation\\\
    Model 5: BCDE as Train, A as Validation \\
    Selection of the best model is obtained by selecting the model with Accuracy and Best Fitting.

    Experiment 3: AdaBoost ensemble with weak learner model taken from experiment 2.

 
    \subsection{The Model Results}
        \subsubsection{Experiment 1: Single Model (YOLOv8 Fine-Tuning)}
        In the first experiment, the YOLOv8 model was trained with data divided in the ratio of 70:20:10 for training, validation, and testing. The dataset consists of:
        Train data: 556 images (70\%)
        Validation Data: 160 images (20\%)
        Test Data: 77 images (10\%)
        
        The model was trained for 100 epochs with a batch size of 16. Figure 1 shows a consistent downward trend in train/box\_loss, train/cls\_loss, and train/dfl\_loss, indicating the model was able to learn effectively. The loss on the validation data also decreased despite slight fluctuations at the beginning of training, which stabilized near the end of training.
        
        Figure 2 displays the Precision and Recall metrics, both of which show significant improvement. The model's Precision stabilized around 0.8, while the Recall reached 0.9, demonstrating the model's ability to correctly classify both fish species.
        
        The model was evaluated using Confusion Matrix (Figure 3), where the model successfully detected O. celebensis with a precision of 0.96 and recall of 0.95, and O. javanicus with a precision of 0.87 and recall of 0.81. Although there were some errors in the classification of O. javanicus, the overall results show that the model was quite effective in this fish species detection and classification task.

        % \begin{figure*}[ht]
        % \centering
        % \includegraphics[scale=0.35]{image/Experiment 1/exp1_box_loss_dashboard.png}
        % \caption{Caption}
        % \label{fig:exp1_box_loss_dashboard}
        % \end{figure*}

        % \begin{figure*}[ht]
        % \centering
        % \includegraphics[width=160mm]{image/Experiment 1/exp1_class_loss_dashboard.png}
        % \caption{Caption}
        % \label{fig:exp1_class_loss_dashboard}
        % \end{figure*}

        % \begin{figure*}[ht]
        % \centering
        % \includegraphics[width=160mm]{image/Experiment 1/exp1_confusion_matrix.png}
        % \caption{Caption}
        % \label{fig:exp1_confusion_matrix}
        % \end{figure*}

        % \begin{figure*}[ht]
        % \centering
        %     \begin{tabular}{cc}
        %         \begin{tabular}[c]{@{}c@{}}
        %             \includegraphics[width=75mm]{image/Experiment 1/exp1_p_curve.png}
        %             \\ a)\end{tabular} & 
        %         \begin{tabular}[c]{@{}c@{}}
        %             \includegraphics[width=75mm]{image/Experiment 1/exp1_r_curve.png}
        %             \\ b)\end{tabular} \\
        %         \begin{tabular}[c]{@{}c@{}}
        %             \includegraphics[width=75mm]{image/Experiment 1/exp1_pr_curve.png}
        %             \\ c)\end{tabular} & 
        %         \begin{tabular}[c]{@{}c@{}}
        %             \includegraphics[width=75mm]{image/Experiment 1/exp1_f1_curve.png}
        %             \\ d)\end{tabular}
        %     \end{tabular}
        % \caption{Caption}
        % \label{fig:exp1_curve}
        % \end{figure*}
        
        % \subsubsection{Experiment 2: 5-Fold Cross Validation}
        % In the second experiment, the 5-fold cross-validation method is applied to evaluate the stability and generalization ability of the YOLOv8 model. The dataset is divided into three main subsets:
        % Train data: 508 images (64\%)
        % Validation Data: 127 images (16\%)
        % Test Data: 158 images (20\%)
        
        % The training process was conducted with a variety of subsets used as validation sets within each fold. Five models were trained using different combinations of data in each fold. The data sharing scheme for each model is as follows:
        % Model 1: Trained on subsets A, B, C, D and validated on subset E
        % Model 2: Trained on subsets A, B, C, E and validated on subset D
        % Model 3: Trained on subsets A, B, D, E and validated on subset C
        % Model 4: Trained on subsets A, C, D, E and validated on subset B
        % Model 5: Trained on subsets B, C, D, E and validated on subset A
        
        % Each cross-validation iteration allows each subset to act as the validation set once, while the other subset is used as the training set. After the 5-fold cross-validation process is complete, the model performance metrics are averaged to obtain a more robust and generalizable performance estimate.
        
        % The cross-validation results show that the average mAP50 value of all folds is 0.78, with an average precision of 0.77 and an average recall of 0.82. The application of the cross-validation method is essential to ensure that the model not only performs well on the training data, but is also able to maintain its performance on unseen data. This proves the model's ability to recognize fish species with a high degree of accuracy outside the training data, ensuring better generalization ability and reducing the risk of overfitting.

        % \begin{figure*}[ht]
        % \centering
        %     \begin{tabular}{cc}
        %         \begin{tabular}[c]{@{}c@{}}
        %             \includegraphics[width=75mm]{image/Experiment 2/exp2_model1_box_loss_dashboard.png}
        %             \\ a)\end{tabular} & 
        %         \begin{tabular}[c]{@{}c@{}}
        %             \includegraphics[width=75mm]{image/Experiment 2/exp2_model2_box_loss_dashboard.png}
        %             \\ b)\end{tabular} \\
        %         \begin{tabular}[c]{@{}c@{}}
        %             \includegraphics[width=75mm]{image/Experiment 2/exp2_model3_box_loss_dashboard.png}
        %             \\ c)\end{tabular} & 
        %         \begin{tabular}[c]{@{}c@{}}
        %             \includegraphics[width=75mm]{image/Experiment 2/exp2_model4_box_loss_dashboard.png}
        %             \\ d)\end{tabular} \\
        %         \multicolumn{2}{c}{\begin{tabular}[c]{@{}c@{}}
        %             \includegraphics[width=75mm]{image/Experiment 2/exp2_model5_box_loss_dashboard.png}
        %             \\ e)\end{tabular}}
        %     \end{tabular}
        % \caption{Caption}
        % \end{figure*}

        % \begin{figure*}[ht]
        % \centering
        %     \begin{tabular}{cc}
        %         \begin{tabular}[c]{@{}c@{}}
        %             \includegraphics[width=75mm]{image/Experiment 2/exp2_model1_class_loss_dashboard.png}
        %             \\ a)\end{tabular} & 
        %         \begin{tabular}[c]{@{}c@{}}
        %             \includegraphics[width=75mm]{image/Experiment 2/exp2_model2_class_loss_dashboard.png}
        %             \\ b)\end{tabular} \\
        %         \begin{tabular}[c]{@{}c@{}}
        %             \includegraphics[width=75mm]{image/Experiment 2/exp2_model3_class_loss_dashboard.png}
        %             \\ c)\end{tabular} & 
        %         \begin{tabular}[c]{@{}c@{}}
        %             \includegraphics[width=75mm]{image/Experiment 2/exp2_model4_class_loss_dashboard.png}
        %             \\ d)\end{tabular} \\
        %         \multicolumn{2}{c}{\begin{tabular}[c]{@{}c@{}}
        %             \includegraphics[width=75mm]{image/Experiment 2/exp2_model5_class_loss_dashboard.png}
        %             \\ e)\end{tabular}}
        %     \end{tabular}
        % \caption{Caption}
        % \end{figure*}

        Based on the results of 5-fold cross-validation and visual analysis on the performance metrics of each model, Model 4 showed the best results with more consistent and higher precision, recall, and mAP values than the other models.

        % \begin{figure*}[ht]
        % \centering
        %     \begin{tabular}{cc}
        %         \begin{tabular}[c]{@{}c@{}}
        %             \includegraphics[width=75mm]{image/Experiment 2/exp2_model1_confusion_matrix.png}
        %             \\ a)\end{tabular} & 
        %         \begin{tabular}[c]{@{}c@{}}
        %             \includegraphics[width=75mm]{image/Experiment 2/exp2_model2_confusion_matrix.png}
        %             \\ b)\end{tabular} \\
        %         \begin{tabular}[c]{@{}c@{}}
        %             \includegraphics[width=75mm]{image/Experiment 2/exp2_model3_confusion_matrix.png}
        %             \\ c)\end{tabular} & 
        %         \begin{tabular}[c]{@{}c@{}}
        %             \includegraphics[width=75mm]{image/Experiment 2/exp2_model4_confusion_matrix.png}
        %             \\ d)\end{tabular} \\
        %         \multicolumn{2}{c}{\begin{tabular}[c]{@{}c@{}}
        %             \includegraphics[width=75mm]{image/Experiment 2/exp2_model5_confusion_matrix.png}
        %             \\ e)\end{tabular}}
        %     \end{tabular}
        % \caption{Caption}
        % \end{figure*}

        % \begin{figure*}[ht]
        % \centering
        % \includegraphics[scale=0.5]{image/Experiment 2/exp2_error_box_loss.png}
        % \caption{Caption}
        % \label{fig:exp2_error_box_loss}
        % \end{figure*}

        % \begin{figure*}[ht]
        % \centering
        % \includegraphics[scale=0.5]{image/Experiment 2/exp2_error_class_loss.png}
        % \caption{Caption}
        % \label{fig:exp2_error_class_loss}
        % \end{figure*}

        
        \subsubsection{Experiment 3: Ensemble Method (AdaBoost)}
        In the third experiment, the ensemble approach was applied by combining five models from the cross-validation results using the AdaBoost algorithm. The aim was to improve the prediction accuracy by correcting the weaknesses of each individual model. Each model was given a different weight based on their performance, with the model that had more prediction errors gaining more weight in subsequent iterations.
        
        After using the AdaBoost ensemble, mAP50 increased to 0.81, and mAP50-95 increased to 0.63. In addition, precision increased to 0.82, while recall increased to 0.86. Combining these models proved to be effective in correcting errors that may occur in individual models, especially in the case of images that are more difficult to identify due to differences in lighting or object position.
        
        In the third experiment, the ensemble approach was applied by combining five models from the cross-validation results using the AdaBoost algorithm. The aim was to improve the prediction accuracy by correcting the weaknesses of each individual model. Each model was given a different weight based on their performance, with the model that had more prediction errors gaining more weight in subsequent iterations.
        
        After using the AdaBoost ensemble, mAP50 increased to 0.81, and mAP50-95 increased to 0.63. In addition, precision increased to 0.82, while recall increased to 0.86. Combining these models proved to be effective in correcting errors that may occur in individual models, especially in the case of images that are more difficult to identify due to differences in lighting or object position.

    \subsection{Performance Evaluation}
    Performance evaluation of the ensemble models showed that combining the five models with the AdaBoost technique resulted in a significant improvement in the detection and classification of both fish species. With an mAP50 of 0.80 and mAP50-95 of 0.65 on the test dataset, this indicates that the ensemble method is highly effective in dealing with real-world image complexity.
    
    The Confusion Matrix of the ensemble model shows an increase in better detection for both species. O. celebensis had a precision of 0.96 and recall of 0.95, while O. javanicus had a precision of 0.87 and recall of 0.81. These results show a steady improvement from each experiment, with the ensemble model giving the best performance.

%%%% Pembahasan %%%%
\section{Discussion}
    \subsection{AdaBoost Ensemble Method}
    One of the main contributions of this research is the application of the AdaBoost ensemble method to the YOLOv8 model. The AdaBoost algorithm helps mitigate the limitations of individual models by focusing on misclassified samples. In each iteration, greater weight is assigned to samples that were incorrectly predicted in the previous iteration, allowing the ensemble to improve detection accuracy for more challenging images, such as those involving small or partially obscured Medaka fish.

    By integrating AdaBoost with YOLOv8, the model becomes more robust, especially in scenarios where the data is imbalanced or complex. This ensemble approach also contributes to reducing bias, enhancing the model's ability to generalize beyond the training data. The improved mAP values, particularly in difficult image scenarios, demonstrate the ensemble's effectiveness in boosting prediction accuracy.
    \subsection{5-Fold Cross-Validation for Model Generalization}
    Another major contribution of this paper is the implementation of 5-fold cross-validation, which played a critical role in preventing overfitting and ensuring that the model performed well on unseen data. By splitting the dataset into five subsets and training the model iteratively, the process provided more reliable and stable performance metrics. Each subset was used once as the validation set while the other four subsets served as training data, resulting in averaged performance metrics that better reflect the model’s true capabilities.

    This method enabled a thorough evaluation of YOLOv8’s performance under different data conditions, showing that the model was capable of detecting and classifying Medaka fish species consistently, even in challenging real-world images. The consistent high precision and recall values across the folds demonstrate the model’s strong generalization ability.
    \subsection{Key Contributions to Real-World Application}
    The integration of these techniques into a YOLOv8-based system for endangered species detection contributes significantly to the field of conservation technology. The ability to detect small, rare species with high accuracy has immediate applications in monitoring populations, tracking habitat changes, and assisting conservation efforts in real-time. AdaBoost and 5-fold cross-validation ensure the model's robustness and reliability in varying environments, offering a practical solution for real-world applications where data might be limited and imbalanced.



Berdasarkan Tabel 2, model P-MobileNetV2 menunjukkan hasil evaluasi kinerja yang lebih baik dibandingkan model lainnya, dengan Sensitivitas = $98.46\%$, Presisi = $98.46\%$, F1 Score = $98.46\%$, dan Akurasi = $98,7\%$. Dalam penelitian ini, kami juga menerapkan teknik padding pada dataset kami. Teknik ini melibatkan penambahan nilai batas buatan untuk mengisi ruang kosong, memastikan gambar tetap berbentuk persegi (rasio aspek 1:1), yang membantu mempertahankan informasi spasial selama proses konvolusi. Hasilnya, model P-MobileNetV2 dan P-VGG16 yang dilatih pada dataset dengan padding mencapai metrik yang lebih tinggi dibandingkan model yang dilatih tanpa padding dataset, membuktikan bahwa teknik ini dapat secara efektif meningkatkan kinerja model. Hal ini dapat diamati pada Gambar 10 dan 12, di mana kedua grafik menunjukkan tren yang secara signifikan lebih baik dibandingkan Gambar 9 dan 11."

Lebih lanjut, penelitian kami memberikan wawasan bahwa teknik padding tidak hanya meningkatkan akurasi model tetapi juga membantu menjaga konsistensi dan mempertahankan konteks spasial pada gambar dataset. Hal ini menjadi sangat krusial untuk tugas-tugas seperti segmentasi dan deteksi objek, di mana informasi spasial dalam gambar atau dataset memiliki tingkat kepentingan yang tinggi. Oleh karena itu, penerapan teknik padding dalam penelitian ini terbukti menjadi langkah kritis dalam meningkatkan kinerja model dan mencapai hasil yang kuat dalam tugas klasifikasi kami.

%%%%%%%%%%%%%%%%%%%%%%%%%%%%%%%%%%%%%%%%%%
%\section{Patents}

%This section is not mandatory, but may be added if there are patents resulting from the work reported in this manuscript.

%%%%%%%%%%%%%%%%%%%%%%%%%%%%%%%%%%%%%%%%%%
\vspace{6pt} 

%%%%%%%%%%%%%%%%%%%%%%%%%%%%%%%%%%%%%%%%%%
%% optional
%\supplementary{The following supporting information can be downloaded at:  \linksupplementary{s1}, Figure S1: title; Table S1: title; Video S1: title.}

% Only for journal Methods and Protocols:
% If you wish to submit a video article, please do so with any other supplementary material.
% \supplementary{The following supporting information can be downloaded at: \linksupplementary{s1}, Figure S1: title; Table S1: title; Video S1: title. A supporting video article is available at doi: link.}

% Only for journal Hardware:
% If you wish to submit a video article, please do so with any other supplementary material.
% \supplementary{The following supporting information can be downloaded at: \linksupplementary{s1}, Figure S1: title; Table S1: title; Video S1: title.\vspace{6pt}\\
%\begin{tabularx}{\textwidth}{lll}
%\toprule
%\textbf{Name} & \textbf{Type} & \textbf{Description} \\
%\midrule
%S1 & Python script (.py) & Script of python source code used in XX \\
%S2 & Text (.txt) & Script of modelling code used to make Figure X \\
%S3 & Text (.txt) & Raw data from experiment X \\
%S4 & Video (.mp4) & Video demonstrating the hardware in use \\
%... & ... & ... \\
%\bottomrule
%\end{tabularx}
%}

\bibliographystyle{apalike} % or IEEEtran, plainnat, etc.
\bibliography{references}   % if your .bib file is named references.bib


\end{document}
