%  LaTeX support: latex@mdpi.com 
%  For support, please attach all files needed for compiling as well as the log file, and specify your operating system, LaTeX version, and LaTeX editor.

%=================================================================
\documentclass[journal,article,submit,pdftex,moreauthors]{Definitions/mdpi}
%\documentclass[preprints,article,submit,pdftex,moreauthors]{Definitions/mdpi} 
% For posting an early version of this manuscript as a preprint, you may use "preprints" as the journal. Changing "submit" to "accept" before posting will remove line numbers.

% Below journals will use APA reference format:
% admsci, aieduc, behavsci, businesses, econometrics, economies, education, ejihpe, famsci, games, humans, ijcs, ijfs, journalmedia, jrfm, languages, psycholint, publications, tourismhosp, youth

% Below journals will use Chicago reference format:
% arts, genealogy, histories, humanities, jintelligence, laws, literature, religions, risks, socsci

%--------------------
% Class Options:
%--------------------
%----------
% journal
%----------
% Choose between the following MDPI journals:
% accountaudit, acoustics, actuators, addictions, adhesives, admsci, adolescents, aerobiology, aerospace, agriculture, agriengineering, agrochemicals, agronomy, ai, air, algorithms, allergies, alloys, amh, analytica, analytics, anatomia, anesthres, animals, antibiotics, antibodies, antioxidants, applbiosci, appliedchem, appliedmath, appliedphys, applmech, applmicrobiol, applnano, applsci, aquacj, architecture, arm, arthropoda, arts, asc, asi, astronomy, atmosphere, atoms, audiolres, automation, axioms, bacteria, batteries, bdcc, behavsci, beverages, biochem, bioengineering, biologics, biology, biomass, biomechanics, biomed, biomedicines, biomedinformatics, biomimetics, biomolecules, biophysica, biosensors, biosphere, biotech, birds, blockchains, bloods, blsf, brainsci, breath, buildings, businesses, cancers, carbon, cardiogenetics, catalysts, cells, ceramics, challenges, chemengineering, chemistry, chemosensors, chemproc, children, chips, cimb, civileng, cleantechnol, climate, clinbioenerg, clinpract, clockssleep, cmd, cmtr, coasts, coatings, colloids, colorants, commodities, complications, compounds, computation, computers, condensedmatter, conservation, constrmater, cosmetics, covid, crops, cryo, cryptography, crystals, csmf, ctn, curroncol, cyber, dairy, data, ddc, dentistry, dermato, dermatopathology, designs, devices, diabetology, diagnostics, dietetics, digital, disabilities, diseases, diversity, dna, drones, dynamics, earth, ebj, ecm, ecologies, econometrics, economies, education, eesp, ejihpe, electricity, electrochem, electronicmat, electronics, encyclopedia, endocrines, energies, eng, engproc, ent, entomology, entropy, environments, epidemiologia, epigenomes, esa, est, famsci, fermentation, fibers, fintech, fire, fishes, fluids, foods, forecasting, forensicsci, forests, fossstud, foundations, fractalfract, fuels, future, futureinternet, futureparasites, futurepharmacol, futurephys, futuretransp, galaxies, games, gases, gastroent, gastrointestdisord, gastronomy, gels, genealogy, genes, geographies, geohazards, geomatics, geometry, geosciences, geotechnics, geriatrics, glacies, grasses, greenhealth, gucdd, hardware, hazardousmatters, healthcare, hearts, hemato, hematolrep, heritage, higheredu, highthroughput, histories, horticulturae, hospitals, humanities, humans, hydrobiology, hydrogen, hydrology, hygiene, idr, iic, ijerph, ijfs, ijgi, ijmd, ijms, ijns, ijpb, ijt, ijtm, ijtpp, ime, immuno, informatics, information, infrastructures, inorganics, insects, instruments, inventions, iot, j, jal, jcdd, jcm, jcp, jcs, jcto, jdad, jdb, jeta, jfb, jfmk, jimaging, jintelligence, jlpea, jmahp, jmmp, jmms, jmp, jmse, jne, jnt, jof, joitmc, joma, jop, jor, journalmedia, jox, jpbi, jpm, jrfm, jsan, jtaer, jvd, jzbg, kidney, kidneydial, kinasesphosphatases, knowledge, labmed, laboratories, land, languages, laws, life, lights, limnolrev, lipidology, liquids, literature, livers, logics, logistics, lubricants, lymphatics, machines, macromol, magnetism, magnetochemistry, make, marinedrugs, materials, materproc, mathematics, mca, measurements, medicina, medicines, medsci, membranes, merits, metabolites, metals, meteorology, methane, metrics, metrology, micro, microarrays, microbiolres, microelectronics, micromachines, microorganisms, microplastics, microwave, minerals, mining, mmphys, modelling, molbank, molecules, mps, msf, mti, multimedia, muscles, nanoenergyadv, nanomanufacturing, nanomaterials, ncrna, ndt, network, neuroglia, neurolint, neurosci, nitrogen, notspecified, nursrep, nutraceuticals, nutrients, obesities, oceans, ohbm, onco, oncopathology, optics, oral, organics, organoids, osteology, oxygen, parasites, parasitologia, particles, pathogens, pathophysiology, pediatrrep, pets, pharmaceuticals, pharmaceutics, pharmacoepidemiology, pharmacy, philosophies, photochem, photonics, phycology, physchem, physics, physiologia, plants, plasma, platforms, pollutants, polymers, polysaccharides, populations, poultry, powders, preprints, proceedings, processes, prosthesis, proteomes, psf, psych, psychiatryint, psychoactives, psycholint, publications, purification, quantumrep, quaternary, qubs, radiation, reactions, realestate, receptors, recycling, regeneration, religions, remotesensing, reports, reprodmed, resources, rheumato, risks, robotics, rsee, ruminants, safety, sci, scipharm, sclerosis, seeds, sensors, separations, sexes, signals, sinusitis, siuj, skins, smartcities, sna, societies, socsci, software, soilsystems, solar, solids, spectroscj, sports, standards, stats, std, stresses, surfaces, surgeries, suschem, sustainability, symmetry, synbio, systems, tae, targets, taxonomy, technologies, telecom, test, textiles, thalassrep, therapeutics, thermo, timespace, tomography, tourismhosp, toxics, toxins, transplantology, transportation, traumacare, traumas, tropicalmed, universe, urbansci, uro, vaccines, vehicles, venereology, vetsci, vibration, virtualworlds, viruses, vision, waste, water, wem, wevj, wild, wind, women, world, youth, zoonoticdis

%---------
% article
%---------
% The default type of manuscript is "article", but can be replaced by: 
% abstract, addendum, article, benchmark, book, bookreview, briefcommunication, briefreport, casereport, changes, clinicopathologicalchallenge, comment, commentary, communication, conceptpaper, conferenceproceedings, correction, conferencereport, creative, datadescriptor, discussion, entry, expressionofconcern, extendedabstract, editorial, essay, erratum, fieldguide, hypothesis, interestingimages, letter, meetingreport, monograph, newbookreceived, obituary, opinion, proceedingpaper, projectreport, reply, retraction, review, perspective, protocol, shortnote, studyprotocol, supfile, systematicreview, technicalnote, viewpoint, guidelines, registeredreport, tutorial,  giantsinurology, urologyaroundtheworld
% supfile = supplementary materials

%----------
% submit
%----------
% The class option "submit" will be changed to "accept" by the Editorial Office when the paper is accepted. This will only make changes to the frontpage (e.g., the logo of the journal will get visible), the headings, and the copyright information. Also, line numbering will be removed. Journal info and pagination for accepted papers will also be assigned by the Editorial Office.

%------------------
% moreauthors
%------------------
% If there is only one author the class option oneauthor should be used. Otherwise use the class option moreauthors.

%---------
% pdftex
%---------
% The option pdftex is for use with pdfLaTeX. Remove "pdftex" for (1) compiling with LaTeX & dvi2pdf (if eps figures are used) or for (2) compiling with XeLaTeX.

%=================================================================
% MDPI internal commands - do not modify
\firstpage{1} 
\makeatletter 
\setcounter{page}{\@firstpage} 
\makeatother
\pubvolume{1}
\issuenum{1}
\articlenumber{0}
\pubyear{2025}
\copyrightyear{2025}
%\externaleditor{Firstname Lastname} % More than 1 editor, please add `` and '' before the last editor name
\datereceived{ } 
\daterevised{ } % Comment out if no revised date
\dateaccepted{ } 
\datepublished{ } 
%\datecorrected{} % For corrected papers: "Corrected: XXX" date in the original paper.
%\dateretracted{} % For retracted papers: "Retracted: XXX" date in the original paper.
\hreflink{https://doi.org/} % If needed use \linebreak
%\doinum{}
%\pdfoutput=1 % Uncommented for upload to arXiv.org
%\CorrStatement{yes}  % For updates
%\longauthorlist{yes} % For many authors that exceed the left citation part

%=================================================================
% Add packages and commands here. The following packages are loaded in our class file: fontenc, inputenc, calc, indentfirst, fancyhdr, graphicx, epstopdf, lastpage, ifthen, float, amsmath, amssymb, lineno, setspace, enumitem, mathpazo, booktabs, titlesec, etoolbox, tabto, xcolor, colortbl, soul, multirow, microtype, tikz, totcount, changepage, attrib, upgreek, array, tabularx, pbox, ragged2e, tocloft, marginnote, marginfix, enotez, amsthm, natbib, hyperref, cleveref, scrextend, url, geometry, newfloat, caption, draftwatermark, seqsplit
% cleveref: load \crefname definitions after \begin{document}

%=================================================================
% Please use the following mathematics environments: Theorem, Lemma, Corollary, Proposition, Characterization, Property, Problem, Example, ExamplesandDefinitions, Hypothesis, Remark, Definition, Notation, Assumption
%% For proofs, please use the proof environment (the amsthm package is loaded by the MDPI class).

%=================================================================
% Full title of the paper (Capitalized)
\Title{MEDAKA-$\tilde{\mathrm{e}}$L: Ensemble Learning for Reliable Detection of Endangered Medaka (Oryzias) Fish}

% MDPI internal command: Title for citation in the left column
\TitleCitation{MEDAKA-$\tilde{\mathrm{e}}$L: Ensemble Learning}

% Author Orchid ID: enter ID or remove command
\newcommand{\orcidauthorA}{0000-0000-0000-000X} % Add \orcidA{} behind the author's name
%\newcommand{\orcidauthorB}{0000-0000-0000-000X} % Add \orcidB{} behind the author's name

% Authors, for the paper (add full first names)
\Author{Rahmatullah R.$^{1}$\orcidA{}, Armin Lawi$^{1,2,3}$, Muhammad Haerul$^{1}$, Iman Mustika Ismail$^{1}$, Irma Andriani$^{4}$, Andi Iqbal Burhanuddin$^{5}$, and Mario K\"oppen$^{6,}$*}

%\longauthorlist{yes}

% MDPI internal command: Authors, for metadata in PDF
\AuthorNames{Rahmatullah R., Armin Lawi, Muhammad Haerul, Iman Mustika Ismail, Irma Andriani, Andi Iqbal Burhanuddin, and Mario K\"oppen}

% MDPI internal command: Authors, for citation in the left column, only choose below one of them according to the journal style
% If this is a Chicago style journal 
% (arts, genealogy, histories, humanities, jintelligence, laws, literature, religions, risks, socsci): 
% Lastname, Firstname, Firstname Lastname, and Firstname Lastname.

% If this is a APA style journal 
% (admsci, behavsci, businesses, econometrics, economies, education, ejihpe, games, humans, ijfs, journalmedia, jrfm, languages, psycholint, publications, tourismhosp, youth): 
% Lastname, F., Lastname, F., \& Lastname, F.

% If this is a ACS style journal (Except for the above Chicago and APA journals, all others are in the ACS format): 
% Lastname, F.; Lastname, F.; Lastname, F.
\isAPAStyle{%
       \AuthorCitation{R., R., Lawi, A., Haerul, M., Ismail, I.M., Andriani, I., Burhanuddin, A.I., \& K\"oppen, M.}
         }{%
        \isChicagoStyle{%
        \AuthorCitation{R., Rahmatullah Test, Armin Lawi, Muhammad Haerul, Iman Mustika Ismail, Irma Andriani, Andi Iqbal Burhanuddin, and Mario K\"oppen.}
        }{
        \AuthorCitation{R., R.; Lawi, A.; Haerul, M.; Ismail, I.M.; Andriani, I.; Burhanuddin, A.I.; K\"oppen, M.}
        }
}

% Affiliations / Addresses (Add [1] after \address if there is only one affiliation.)
\address{%
$^{1}$ \quad Information Systems Study Program, Faculty of Mathematics and Natural Sciences, Hasanuddin University, Makassar 90245, Indonesia; rahmatullah@unhas.ac.id (R.R.); armin@unhas.ac.id (A.L.); haerul@unhas.ac.id (M.H.); imanmustika@unhas.ac.id (I.M.I.)\\
$^{2}$ \quad Data Science and Artificial Intelligence Research Group, Hasanuddin University, Makassar 90245, Indonesia\\
$^{3}$ \quad B.J. Habibie Institute of Technology, Parepare 91132, Indonesia\\
$^{4}$ \quad Department of Biology, Faculty of Mathematics and Natural Sciences, Hasanuddin University, Makassar 90245, Indonesia; irma.andriani@unhas.ac.id\\
$^{5}$ \quad Department of Fishery, Faculty of Marine Science and Fisheries, Hasanuddin University, Makassar 90245, Indonesia; andi.iqbal@unhas.ac.id\\
$^{6}$ \quad Graduate School of Life Science and Systems Engineering, Kyushu Institute of Technology, Kitakyushu 808-0196, Japan}

% Contact information of the corresponding author
\corres{Correspondence: mkoeppen@brain.kyutech.ac.jp; Tel.: +81-93-884-3225 (M.K.)}

% Current address and/or shared authorship
%\firstnote{Current address: Affiliation.}  
% Current address should not be the same as any items in the Affiliation section.

%\secondnote{These authors contributed equally to this work.}
% The commands \thirdnote{} till \eighthnote{} are available for further notes.

%\simplesumm{} % Simple summary

%\conference{} % An extended version of a conference paper

% Abstract (Do not insert blank lines, i.e. \\) 
\abstract{Medaka (Oryzias) fish, such as the Java medaka (Oryzias javanicus) and Celebes medaka (Oryzias celebensis), play important roles in biodiversity and serve as bioindicators of aquatic ecosystem health in Indonesia. Despite their ecological value, detection remains challenging due to the lack of a dedicated dataset, subtle morphological differences, and environmental variability. To address this, we introduce the first manually annotated Medaka dataset containing 1,280 images and evaluate YOLOv8 against an ensemble using Weighted Boxes Fusion (WBF). The ensemble model achieved a mean mAP@0.5:0.95 that increased from 0.4600 to 0.5571, an 18.6\% improvement in bounding box tightness, while the F1-score improved from 0.5915 to 0.7309, representing a 23.6\% gain in detection accuracy. Although inference speed decreased by approximately 4.3x compared to single models, the accuracy gains highlight the ensemble's value for developing reliable, non-invasive detection systems for ecological monitoring and conservation of endangered Medaka populations.}

% Keywords
\keyword{YOLOv8; object detection; ensemble learning; weighted boxes fusion; non-maximum suppression; ecological monitoring} 

% The fields PACS, MSC, and JEL may be left empty or commented out if not applicable
%\PACS{J0101}
%\MSC{}
%\JEL{}

%%%%%%%%%%%%%%%%%%%%%%%%%%%%%%%%%%%%%%%%%%
% Only for the journal Diversity
%\LSID{\url{http://}}

%%%%%%%%%%%%%%%%%%%%%%%%%%%%%%%%%%%%%%%%%%
% Only for the journal Applied Sciences
%\featuredapplication{Authors are encouraged to provide a concise description of the specific application or a potential application of the work. This section is not mandatory.}
%%%%%%%%%%%%%%%%%%%%%%%%%%%%%%%%%%%%%%%%%%

%%%%%%%%%%%%%%%%%%%%%%%%%%%%%%%%%%%%%%%%%%
% Only for the journal Data
%\dataset{DOI number or link to the deposited data set if the data set is published separately. If the data set shall be published as a supplement to this paper, this field will be filled by the journal editors. In this case, please submit the data set as a supplement.}
%\datasetlicense{License under which the data set is made available (CC0, CC-BY, CC-BY-SA, CC-BY-NC, etc.)}

%%%%%%%%%%%%%%%%%%%%%%%%%%%%%%%%%%%%%%%%%%
% Only for the journal BioTech, Fishes, Neuroimaging and Toxins
%\keycontribution{The breakthroughs or highlights of the manuscript. Authors can write one or two sentences to describe the most important part of the paper.}

%%%%%%%%%%%%%%%%%%%%%%%%%%%%%%%%%%%%%%%%%%
% Only for the journal Encyclopedia
%\encyclopediadef{For entry manuscripts only: please provide a brief overview of the entry title instead of an abstract.}

%%%%%%%%%%%%%%%%%%%%%%%%%%%%%%%%%%%%%%%%%%
% Only for the journal Advances in Respiratory Medicine, Future, Sensors and Smart Cities
%\addhighlights{yes}
%\renewcommand{\addhighlights}{%
%
%\noindent This is an obligatory section in ``Advances in Respiratory Medicine'', ``Future'', ``Sensors'' and ``Smart Cities", whose goal is to increase the discoverability and readability of the article via search engines and other scholars. Highlights should not be a copy of the abstract, but a simple text allowing the reader to quickly and simplified find out what the article is about and what can be cited from it. Each of these parts should be devoted up to 2~bullet points.\vspace{3pt}\\
%\textbf{What are the main findings?}
% \begin{itemize}[labelsep=2.5mm,topsep=-3pt]
% \item First bullet.
% \item Second bullet.
% \end{itemize}\vspace{3pt}
%\textbf{What is the implication of the main finding?}
% \begin{itemize}[labelsep=2.5mm,topsep=-3pt]
% \item First bullet.
% \item Second bullet.
% \end{itemize}
%}

%%%%%%%%%%%%%%%%%%%%%%%%%%%%%%%%%%%%%%%%%%
\begin{document}

%%%%%%%%%%%%%%%%%%%%%%%%%%%%%%%%%%%%%%%%%%
\section{Introduction}

Medaka fish (Oryzias) are small freshwater fish valued as ornamental species and are significant for biodiversity studies. According to the International Union for Conservation of Nature (IUCN), medaka fish species are classified as vulnerable, which makes protecting them an ecological priority. However, identifying medaka fish apart is challenging since they have subtle morphological differences and small sizes. Not only that, but the variability of aquatic environments also makes it even more challenging. Traditionally, medaka fish have been sacrificed in taxonomy studies due to their genetic value for lineage studies ~\cite{Mahmudi2022}. Their body colour serves as a social signal and reflects environmental conditions~\cite{RapidBodyColourationOryzias2020}, making them an indicator of ecosystem health. Traditional research approaches involving capturing and sacrificing specimens for anatomical or genetic examination are inherently invasive and unsuitable for sustained ecological monitoring, can harm the organism and the natural environment, and thus limit their application in long-term ecological and genetic studies. Reliable detection technologies can facilitate non-invasive conservation without the need for traditional conservation methods. This technique protects species and ecosystems while enabling effective ecological monitoring in conservation efforts.

Deep learning has evolved beyond digit identification to sophisticated object detection, facilitating applications in autonomous vehicles, medical diagnostics, agricultural automation, and environmental monitoring~\cite{LeCun2015,Zhao2019}.In ecological research, object detection has been utilized for monitoring insects in agriculture~\cite{Tang2023,Ciampi2023} and detecting wildlife in natural habitats~\cite{Roy2023,Wenhan2024,Sun2024}, where precise identification is frequently challenging due to visual similarities among species and the complexity of water environments. These issues are also present in the medaka fish species, such as Oryzias javanicus and Oryzias celebensis, which have subtle morphological distinctions that complicate the reliable identification of endemic fish.

The YOLO architecture is one of the most widely used deep learning-based one-stage object detectors. Among its versions, YOLOv8 has shown a particularly good balance between speed and accuracy, as demonstrated in comparative benchmarks that evaluated YOLOv8 through YOLOv11 under real-world conditions~\cite{Sapkota2024}. Researchers have also proposed many variants to push YOLO’s performance further, especially in challenging settings. For example, CEH-YOLO adds a high-order deformable attention (HDA) module to better highlight important spatial features, an Enhanced Spatial Pyramid Pooling-Fast (ESPPF) module for improved texture and color feature extraction, and a Composite Detection module to boost detection of small or overlapping underwater objects, along with using WIoU-v3 loss to improve bounding box regression under hard conditions~\cite{CEH-YOLO}. ~\cite{YOLO-SAG} introduces the Softplus activation function to improve training stability, an AIFI module to strengthen intra-scale feature interactions (reducing false positives and missed detections), and lightweight neck convolution modules (GSConv, VoV-GSCSP) to reduce computational overhead while maintaining accuracy. SCoralDet focuses on underwater soft coral detection, using a Multi-Path Fusion Block (MPFB) to handle varied scales and lighting colors, lightweight modules for efficiency, and an Adaptive Power Transformation label assignment strategy to better align anchors with ground truth when coral structures are complex or blurred~\cite{SCoralDet}. While these studies mainly alter the internal YOLO architecture to address trade-offs between accuracy and speed, in our work, we propose the use of an ensemble method that builds on YOLOv8 without modifying its core architecture by combining the strengths of multiple detection heads or models to improve reliability under environmental variability.

Despite its strong performance, YOLO models could still make misclassifications by detecting background regions as objects or producing duplicate overlapping predictions of the same object. By default, these redundant outputs are reduced through Non-Maximum Suppression (NMS), yet this technique has well-known limitations, particularly when objects overlap or when multiple plausible predictions exist. Recent studies have come up with a new ensemble method, Weighted Boxes Fusion (WBF), that provides a more accurate and robust alternative, since it merges bounding boxes based on confidence scores and spatial alignment rather than discarding valuable predictions outright~\cite{Solovyev2021}. The limitations of standard non-maximum suppression (NMS) underscore the need to investigate advanced fusion strategies in object detection. Ensemble methods are particularly effective because they combine multiple models or detection heads, thereby improving uncertainty management, reducing false detections, and increasing consistency in challenging scenarios. The primary contribution of this work is the development of an ensemble approach for YOLOv8 that incorporates Weighted Boxes Fusion (WBF) to improve detection reliability in ecological monitoring applications, where accuracy is essential and misclassification significantly affect conservation outcomes.

In this work, we propose a novel Medaka fish dataset with images captured under diverse lighting colors to provide a realistic and challenging benchmark for ecological monitoring. Building on this resource MEDAKA-$\tilde{\mathrm{e}}$L, an ensemble-based detection framework for accurate and reliable identification of Medaka fish. By training 5 different YOLOv8n models trained across a 5-fold cross-validation of the dataset, which causes the model to learn its own unique feature of the Medaka fish.  At the ensemble stage, predictions from multiple YOLOv8 models trained through cross-validation are combined using Weighted Boxes Fusion (WBF). It addresses challenges such as background misclassifications and redundant overlapping bounding boxes. Unlike traditional Non-Maximum Suppression (NMS), WBF merges bounding boxes based on confidence scores and spatial alignment, preserving valuable detections and reducing false positives that achieve higher accuracy. The key contributions of this work are as follows:

\begin{itemize}
    \item Introduced the MEDAKA-$\tilde{\mathrm{e}}$L framework that uses 5 models trained across 5-fold cross-validation, and combines their outputs with Weighted Boxes Fusion (WBF) to improve detection accuracy.

    \item Proposed the Medaka dataset, a new collection of Medaka fish images that includes manually captured photos under different lighting colors, as well as annotated samples gathered from the internet.

    \item Developed an ensemble detection approach based on YOLOv8n models, where predictions are merged with WBF to keep valuable detections and reduce false positives.
\end{itemize}

% The remainder of this manuscript is structured as follows: Section~2 provides a comprehensive literature review encompassing object detection architectures, ensemble learning methodologies, and aquatic species monitoring applications. Section~3 details the proposed methodology including dataset preparation protocols, model architecture specifications, and ensemble fusion technique implementations. Section~4 presents extensive experimental results with detailed performance analysis and statistical validation. Section~5 discusses the broader implications of findings, acknowledges limitations, and explores practical deployment considerations. Finally, Section~6 concludes with future research directions and potential extensions of this work.

%%%%%%%%%%%%%%%%%%%%%%%%%%%%%%%%%%%%%%%%%%

\subsection{Ensemble Learning in Object Detection}

Ensemble learning principles, originally developed for classification tasks~\cite{Dietterich2000}, have been successfully adapted to object detection scenarios with unique challenges and opportunities. The fundamental premise of ensemble learning—that combining multiple diverse models can achieve superior performance compared to individual models—applies particularly well to detection tasks where model diversity can capture complementary aspects of object appearance and spatial relationships~\cite{Zhou2002}.

Traditional ensemble approaches in classification, including bagging~\cite{Breiman2001} and boosting~\cite{Freund1997}, have been extended to detection scenarios, though the integration of spatial predictions introduces additional complexity. The challenge of combining multiple bounding box predictions from different models has led to specialized fusion techniques beyond simple voting mechanisms used in classification ensembles.

Recent work has explored various ensemble strategies specifically for object detection. These approaches range from simple averaging of confidence scores to sophisticated fusion techniques that consider spatial relationships between predictions. The choice of ensemble strategy significantly impacts both detection accuracy and computational efficiency, requiring careful consideration of application-specific requirements.

\subsection{Bounding Box Fusion Techniques}

The fusion of bounding box predictions from multiple models represents a critical component of ensemble object detection systems. Traditional Non-Maximum Suppression (NMS) operates by selecting the highest-confidence detection and suppressing nearby overlapping detections based on intersection-over-union (IoU) thresholds. While effective for single-model scenarios, NMS may not optimally handle the diverse prediction landscapes generated by ensemble systems.

Weighted Boxes Fusion (WBF)~\cite{Solovyev2021} emerged as an advanced alternative to NMS, specifically designed for ensemble scenarios. Rather than suppressing overlapping boxes, WBF intelligently merges predictions by computing weighted averages of bounding box coordinates and confidence scores. This approach considers both the confidence of individual predictions and their spatial relationships, potentially preserving valuable information that would be discarded by traditional NMS approaches.

The WBF algorithm operates by clustering nearby predictions based on IoU overlap, then computing weighted averages of coordinates and confidences within each cluster. This approach has demonstrated superior performance in various ensemble detection scenarios, particularly when dealing with overlapping objects or uncertain boundaries.

\subsection{Aquatic Species Detection and Monitoring}

The application of computer vision techniques to aquatic species monitoring represents a rapidly growing field with significant ecological and conservation implications. Underwater imaging presents unique challenges including variable lighting conditions, water turbidity, complex backgrounds, and distortions introduced by water medium effects~\cite{Kalafi2018,Leow2015}.

Early approaches to automated fish detection relied on traditional computer vision techniques, including background subtraction and handcrafted feature extraction. However, these methods struggled with the complexity and variability of underwater environments, leading to limited practical adoption in field monitoring scenarios.

Deep learning approaches have shown considerable promise for aquatic species detection and classification. Qin et al.~\cite{Qin2016} developed DeepFish, one of the first deep learning systems specifically designed for underwater fish recognition, demonstrating the potential of CNNs for this application domain. Subsequent work has explored various architectures and training strategies for improved performance in challenging underwater conditions.

Recent advances have focused on addressing specific challenges in aquatic monitoring, including species classification~\cite{Tamou2021}, behavioral analysis~\cite{Salimi2016}, and population assessment~\cite{Mandal2018}. These systems have demonstrated practical utility in ecological research and conservation applications, though challenges remain in achieving the accuracy and reliability required for large-scale deployment.
\subsection{Cross-Validation and Model Evaluation}

Robust evaluation methodologies are essential for assessing the performance and generalizability of detection systems. Cross-validation techniques, originally developed for classification tasks~\cite{Stone1974,Kohavi1995}, have been adapted for object detection scenarios with modifications to account for spatial prediction requirements.

K-fold cross-validation provides a systematic approach to assess model performance across different data splits, helping to identify overfitting and ensure generalization to unseen data~\cite{Browne2000}. In detection tasks, careful consideration must be given to maintaining class balance and spatial distribution across folds.

The COCO evaluation protocol has emerged as the standard for object detection assessment, providing comprehensive metrics including mean Average Precision (mAP) across different IoU thresholds and object scales. These metrics enable detailed analysis of detection performance across various scenarios and facilitate meaningful comparisons between different approaches.

\subsection{Data Augmentation Strategies}

Data augmentation has proven essential for training robust detection models, particularly in scenarios with limited training data or high environmental variability~\cite{Shorten2019}. Augmentation techniques for object detection must carefully preserve spatial relationships between objects and their bounding boxes while introducing appropriate variations to improve generalization.

Common augmentation strategies include geometric transformations (rotation, scaling, translation), photometric adjustments (brightness, contrast, color variation), and advanced techniques such as mixup and cutout. The selection and parameterization of augmentation strategies significantly impacts model performance and requires careful consideration of domain-specific characteristics.

In aquatic imaging scenarios, specific augmentation strategies may be particularly relevant, including simulation of water distortion effects, lighting colors variations, and turbidity changes. These domain-specific augmentations can improve model robustness to the challenging conditions encountered in real-world aquatic monitoring applications.

%%%%%%%%%%%%%%%%%%%%%%%%%%%%%%%%%%%%%%%%%%
\section{Related Work}

\subsection{Object Detection Architectures}

The evolution of object detection has been marked by several paradigm shifts, beginning with traditional computer vision approaches and progressing to sophisticated deep learning architectures. Early detection systems relied on handcrafted features and classical machine learning techniques, exemplified by the Viola-Jones framework~\cite{Freund1997}, which introduced the concept of boosting for object detection applications.

The advent of deep learning revolutionized object detection through the introduction of region-based approaches. R-CNN~\cite{Girshick2014} pioneered the integration of CNNs for feature extraction in detection pipelines, though computational efficiency remained a significant limitation. Subsequent developments including Fast R-CNN and Faster R-CNN~\cite{Ren2015} addressed these efficiency concerns while maintaining high detection accuracy. The Cascade R-CNN architecture~\cite{Cai2018} further refined this approach by implementing progressive refinement of detection quality through multiple detection stages.

Single-shot detection methods emerged as a response to the computational demands of region-based approaches. The Single Shot MultiBox Detector (SSD)~\cite{Liu2019} and the YOLO family~\cite{Redmon2016,Redmon2017} demonstrated that competitive accuracy could be achieved while maintaining real-time processing capabilities. YOLOv4~\cite{Bochkovskiy2020} and subsequent iterations have continued to push the boundaries of this efficiency-accuracy trade-off.

The latest YOLOv8 architecture represents the current state-of-the-art in real-time object detection, incorporating advanced features including anchor-free detection, enhanced feature pyramid networks, and optimized training procedures~\cite{Terven2023}. These improvements have resulted in significant performance gains across diverse detection benchmarks while maintaining computational efficiency suitable for deployment in resource-constrained environments.

\subsection{Ensemble Learning in Object Detection}

Ensemble learning principles, originally developed for classification tasks~\cite{Dietterich2000}, have been successfully adapted to object detection scenarios with unique challenges and opportunities. The fundamental premise of ensemble learning—that combining multiple diverse models can achieve superior performance compared to individual models—applies particularly well to detection tasks where model diversity can capture complementary aspects of object appearance and spatial relationships~\cite{Zhou2002}.
@@ -268,15 +243,6 @@ \subsection{Aquatic Species Detection and Monitoring}
Deep learning approaches have shown considerable promise for aquatic species detection and classification. Qin et al.~\cite{Qin2016} developed DeepFish, one of the first deep learning systems specifically designed for underwater fish recognition, demonstrating the potential of CNNs for this application domain. Subsequent work has explored various architectures and training strategies for improved performance in challenging underwater conditions.

Recent advances have focused on addressing specific challenges in aquatic monitoring, including species classification~\cite{Tamou2021}, behavioral analysis~\cite{Salimi2016}, and population assessment~\cite{Mandal2018}. These systems have demonstrated practical utility in ecological research and conservation applications, though challenges remain in achieving the accuracy and reliability required for large-scale deployment.

\subsection{Medaka Fish as Model Organisms}

Medaka fish (Oryzias species) have gained prominence as important model organisms in both laboratory research and ecological monitoring contexts. These small freshwater fish are widely distributed across Asian aquatic ecosystems and exhibit characteristics that make them valuable for biodiversity studies and environmental monitoring programs.

The morphological similarity between different Oryzias species presents particular challenges for automated detection and classification systems. Traditional identification requires expert knowledge and careful examination of subtle morphological features, making automated approaches particularly valuable for large-scale monitoring efforts.

Previous work on Medaka detection has primarily focused on laboratory settings with controlled imaging conditions. The extension to natural environments with variable lighting, backgrounds, and water conditions represents a significant technical challenge that has not been thoroughly addressed in existing literature.

\subsection{Cross-Validation and Model Evaluation}

Robust evaluation methodologies are essential for assessing the performance and generalizability of detection systems. Cross-validation techniques, originally developed for classification tasks~\cite{Stone1974,Kohavi1995}, have been adapted for object detection scenarios with modifications to account for spatial prediction requirements.
@@ -291,12 +257,14 @@ \subsection{Data Augmentation Strategies}

Common augmentation strategies include geometric transformations (rotation, scaling, translation), photometric adjustments (brightness, contrast, color variation), and advanced techniques such as mixup and cutout. The selection and parameterization of augmentation strategies significantly impacts model performance and requires careful consideration of domain-specific characteristics.

In aquatic imaging scenarios, specific augmentation strategies may be particularly relevant, including simulation of water distortion effects, lighting variations, and turbidity changes. These domain-specific augmentations can improve model robustness to the challenging conditions encountered in real-world aquatic monitoring applications.

%%%%%%%%%%%%%%%%%%%%%%%%%%%%%%%%%%%%%%%%%%
\section{Materials and Methods}

\subsection{Hybrid Dataset Construction and Curation}

Given the scarcity of publicly available Medaka fish imagery and the critical need for taxonomic precision in ecological monitoring, we developed a comprehensive hybrid dataset that strategically combines primary field collection with curated internet-sourced imagery. This approach addresses the fundamental challenge identified in our introduction regarding the lack of dedicated datasets for endangered Medaka species detection, while ensuring sufficient morphological diversity for robust model training.

Our final dataset comprises 1,139 unique images containing 1,280 annotated Medaka (\textit{Oryzias}) instances, systematically acquired through a dual-source methodology. The dataset encompasses two endemic Indonesian species of conservation value: 641 instances of \textit{Oryzias celebensis} and 639 instances of \textit{Oryzias javanicus}, a balanced distribution that supports unbiased model learning of inter-species differences.

\subsubsection{Primary Data Acquisition Protocol}
Primary acquisition combined (i) controlled laboratory imaging (Genetics Laboratory, Faculty of Mathematics and Natural Sciences, Hasanuddin University) and (ii) in-situ observation in the Tanjung coastal freshwater-estuarine transition zone (Makassar). The dual setting was selected to pair morphology-focused clarity with authentic environmental variability needed for generalizable detection.

In the laboratory, \textit{O. celebensis} and \textit{O. javanicus} were photographed in modular glass aquaria with four background colors (red, black, blue, green) to (1) enhance contrast across pigmentation states, (2) avoid overfitting to a single chromatic/illumination pairing, and (3) approximate substrate tone diversity. Diffuse LED illumination minimized glare and motion shadows while preserving natural coloration.

Field imagery contributed heterogeneous backgrounds (irregular substrates, suspended particulates, natural illumination shifts). Frames with severe turbidity, strong specular glare, heavy occlusion, or indistinct body outlines were discarded. Near-duplicate temporal sequences, excessive motion blur, and pronounced chromatic aberration were also removed. Only light normalization (orientation alignment and rejection of unusable frames) was applied—no denoising or color correction—to retain genuine variance the ensemble can exploit.

This streamlined protocol operationalizes the Introduction's highlighted challenges (subtle inter-species morphology and aquatic visual variability) by producing complementary distributions: laboratory images supply structural clarity; field images supply natural clutter and illumination variation. The resulting diversity yields moderately decorrelated bounding box hypotheses across cross-validation folds, improving the effectiveness of subsequent Weighted Boxes Fusion (WBF) without relying on synthetic augmentation artifacts.

\subsubsection{Internet-Sourced Data Curation and Validation}

Secondary data acquisition involved systematic curation of internet-available imagery from scientific databases, aquaculture forums, and research repositories. This process employed automated web scraping techniques supplemented by manual validation to ensure taxonomic accuracy and image quality standards. Search protocols targeted specialized ichthyological databases, university research collections, and peer-reviewed publications containing Medaka imagery.

Each internet-sourced image underwent rigorous quality assessment including: (1) taxonomic verification by expert ichthyologists, (2) resolution adequacy analysis (minimum 800×600 pixels), (3) morphological feature visibility assessment, and (4) metadata completeness evaluation. Images displaying taxonomic ambiguity, poor resolution, or significant distortion were excluded from the final dataset. This curation process resulted in a 68\% acceptance rate from initially collected internet sources, ensuring dataset quality while maximizing morphological diversity.

\subsubsection{Dataset Composition and Statistical Characteristics}
The final dataset comprises 1,139 unique images containing 1,280 manually annotated Medaka instances across two target species: \textit{O. celebensis} (n=641; 50.1\% of instances) and \textit{O. javanicus} (n=639; 49.9\% of instances). Because multiple fish may appear in a single frame, annotated instances exceed image count. Image sources include 955 primary (laboratory + field) images (83.9\%) and 184 curated internet images (16.1\%). This sourcing mix preserves a majority of controlled-quality imagery while injecting environmental and background diversity required for robust generalization.

Images have an average native resolution of 2847×2134 pixels (SD 892×634), delivering ample spatial detail to represent fine small-body morphology; more than 95\% exceed 1 MP. Only light preprocessing was applied, preserving natural chromatic variation, with 97.3\% of samples remaining within acceptable color gamut limits. This combination of high spatial fidelity and retained color realism supports the model in learning subtle inter-species differences essential to the detection task. Representative visual diversity in species, backgrounds, and acquisition conditions is shown in Figure~\ref{fig:dataset-samples}.

\begin{figure}[H]
%\isPreprints{\centering} % If the paper is ``preprints'', please uncomment this parenthesis.
		\centering
		\includegraphics[width=0.8\textwidth]{Images/mixture-medaka-dataset.pdf}
%} % If the paper is ``preprints'', please uncomment this parenthesis.
\caption{Representative samples from the Medaka dataset showing diversity in species, environmental conditions, and imaging scenarios. (\textbf{a}) \textit{O. celebensis} specimens in various naturalistic settings. (\textbf{b}) \textit{O. javanicus} specimens demonstrating morphological variation and environmental diversity.\label{fig:dataset-samples}}
	\end{adjustwidth}
\end{figure}

\subsubsection{Ground Truth Annotation and Quality Assurance}

All images were manually annotated using the Roboflow platform to generate YOLO‑format bounding boxes for each visible fish instance. A simple, consistent guideline was followed: each box closely encloses the fish body with a small uniform margin (approximately 10\% padding) to ensure complete coverage of fins without excessive background. Two class labels were used (\textit{O. celebensis}, \textit{O. javanicus}) and applied based on readily observable external morphology (overall body shape, fin outline, and pigmentation pattern) without relying on specialized taxonomic dissection.

To promote consistency, an internal checklist was applied during annotation: (i) confirm the fish is fully visible (or clearly distinguishable if partial), (ii) avoid truncating fins, (iii) exclude reflections or duplicate shadows, and (iv) skip ambiguous shapes lacking clear anatomical structure. If multiple fish appeared in a frame, each was annotated separately; overlapping individuals received distinct boxes.

Quality assurance focused on basic internal consistency rather than formal expert validation metrics. A subset of 247 images (21.7\% of the dataset) was rechecked manually: boxes with obvious overreach (large empty margins) or under-coverage (cropping fins or tail) were adjusted, and uncertain species labels were set aside for later clarification or removed if unverifiable.

This lightweight but systematic review ensured that bounding boxes are well-aligned and labels are applied conservatively, providing sufficiently clean ground truth for training the 5-fold YOLOv8 models and enabling effective fusion under the WBF ensemble framework.

\subsubsection{Dataset Partitioning and Evaluation Framework}

The dataset of 1,139 images was partitioned using 5-fold stratified cross-validation to evaluate individual YOLOv8 models and their ensemble without relying on a single train/validation split. Because 1,139 is not divisible by five, folds contained either 228 or 227 images (three folds with 228, two with 227). For each iteration, one fold served as validation ($\approx19.9\%$), and the remaining four folds ($\approx80.1\%$; 911–912 images depending on the held-out fold) formed the training set.

Stratification preserved: (i) near-equal species proportions (50.1\% \textit{O. celebensis}, 49.9\% \textit{O. javanicus}); (ii) source balance (primary vs. curated internet); and (iii) diversity in background/lighting conditions. This ensured every validation fold reflected the overall distribution rather than an easier or harder subset.

Evaluation followed COCO-style metrics: mAP@0.5:0.95 as the primary indicator, plus mAP@0.5, precision, recall, and F1-score. These were computed per fold and then averaged to summarize central performance trends. For ensemble fusion comparison (NMS vs. WBF), we additionally examined redundant detection counts and confidence score consolidation effects, allowing qualitative inspection of how fusion affected duplicate box suppression or merging. No additional derived statistical tests are reported in this subsection; detailed comparative results appear in the Results section.

\subsection{Data Augmentation Strategy}

To enhance model robustness and generalization capability, we implemented a comprehensive data augmentation pipeline specifically designed for aquatic imaging scenarios. The augmentation strategy encompasses both geometric and photometric transformations while preserving the spatial integrity of bounding box annotations.

\begin{figure}[H]
%\isPreprints{\centering} % If the paper is ``preprints'', please uncomment this parenthesis.
		\centering
		\includegraphics[width=0.45\textwidth]{Images/augmentation1.png}
		\includegraphics[width=0.45\textwidth]{Images/augmentation2.png}
%} % If the paper is ``preprints'', please uncomment this parenthesis.
\caption{Data augmentation examples demonstrating the range of transformations applied to enhance dataset diversity. (\textbf{a}) Original image with ground truth annotations. (\textbf{b}) Augmented versions showing geometric transformations, photometric adjustments, and simulated aquatic distortion effects.\label{fig:data-augmentation}}
	\end{adjustwidth}
\end{figure}

Geometric augmentations include random rotation (±15°), horizontal flipping (probability 0.5), scaling (0.8-1.2×), and translation (±10\% of image dimensions). Photometric augmentations encompass brightness adjustment (±20\%), contrast variation (±15\%), hue shifting (±10°), and saturation modification (±20\%). Additionally, we incorporated domain-specific augmentations including Gaussian noise injection ($\sigma$ = 0-0.05), simulated water ripple effects, and varying degrees of motion blur to replicate realistic underwater imaging conditions.

The augmentation pipeline increased the effective training dataset size by a factor of 8×, resulting in approximately 10,000 training instances per fold during cross-validation. This expansion significantly enhanced the model's ability to generalize across diverse environmental conditions and imaging scenarios.

\subsection{YOLOv8 Base Architecture}

We adopted YOLOv8 as our foundational detection architecture due to its superior balance of accuracy and computational efficiency. YOLOv8 incorporates several architectural innovations including anchor-free detection, enhanced feature pyramid networks (FPN), and optimized activation functions that contribute to improved detection performance.

\begin{figure}[H]
%\isPreprints{\centering} % If the paper is ``preprints'', please uncomment this parenthesis.
		\centering
		\includegraphics[width=0.8\textwidth]{Images/yolov8-architecture.jpg}
%} % If the paper is ``preprints'', please uncomment this parenthesis.
\caption{YOLOv8 architecture overview showing the backbone network, feature pyramid structure, and detection heads. The architecture employs anchor-free detection with multiple prediction scales to handle objects of varying sizes effectively.\label{fig:yolov8-architecture}}
	\end{adjustwidth}
\end{figure}

The YOLOv8 backbone utilizes a modified CSPDarknet architecture with efficient cross-stage partial connections and spatial pyramid pooling. The feature pyramid network enables multi-scale feature extraction and fusion, facilitating detection of objects across different size ranges. The detection head employs decoupled architectures for classification and localization tasks, improving convergence and final performance.

Model training was conducted using the AdamW optimizer with an initial learning rate of 0.001, weight decay of 0.0005, and cosine annealing learning rate scheduling. Training proceeded for 300 epochs with early stopping based on validation mAP monitoring. Input images were resized to 640×640 pixels while maintaining aspect ratios through appropriate padding to preserve spatial relationships.

\subsection{K-Fold Cross-Validation Protocol}

To ensure robust performance evaluation and minimize bias associated with specific train-test splits, we implemented a systematic 5-fold cross-validation protocol. The dataset was stratified based on species distribution and imaging conditions to maintain representative distributions across all folds.

\begin{figure}[H]
%\isPreprints{\centering}{} % Only used for preprints
\includegraphics[width=0.6\textwidth]{Images/kfold-illustration.png}
\caption{Illustration of the 5-fold cross-validation strategy employed for model training and evaluation. Each fold maintains balanced species representation and environmental diversity to ensure robust performance assessment.\label{fig:kfold-method}}
\end{figure}

Each fold consisted of approximately 1,024 training images and 256 validation images (standard 80/20 split of the 1,280-image dataset), with careful attention to maintaining species balance and environmental diversity within each partition. This stratification approach ensures that each model encounters the full range of morphological and environmental variations present in the dataset.

The cross-validation protocol generated five independent YOLOv8 models, each trained on a different 80\% subset of the data and validated on the remaining 20\%. This approach provides robust estimates of model performance while enabling ensemble construction from complementary models trained on overlapping but distinct data distributions.

\subsection{Ensemble Framework Implementation}

Our ensemble framework combines predictions from the five cross-validation models using two distinct fusion strategies: traditional Non-Maximum Suppression (NMS) and advanced Weighted Boxes Fusion (WBF). This comparative approach enables systematic evaluation of fusion strategy effectiveness in ensemble detection scenarios.

\begin{figure}[H]
%\isPreprints{\centering} % If the paper is ``preprints'', please uncomment this parenthesis.
		\centering
		\includegraphics[width=0.45\textwidth]{Images/ensemble-nms-architecture.png}
		\includegraphics[width=0.45\textwidth]{Images/wbf-architecture-1.png}
%} % If the paper is ``preprints'', please uncomment this parenthesis.
\caption{Ensemble architecture comparison: (\textbf{a}) NMS-based ensemble pipeline showing traditional suppression of overlapping predictions. (\textbf{b}) WBF-based ensemble demonstrating intelligent fusion through weighted averaging of spatially related predictions.\label{fig:ensemble-architectures}}
	\end{adjustwidth}
\end{figure}

The NMS ensemble approach aggregates all predictions from the five models, then applies traditional non-maximum suppression with IoU threshold of 0.5 and confidence threshold tuning. This baseline approach provides a reference point for ensemble performance using established techniques.

The WBF ensemble implements the advanced weighted boxes fusion algorithm~\cite{Solovyev2021}, which clusters spatially overlapping predictions and computes weighted averages of coordinates and confidence scores. The WBF implementation uses intersection threshold of 0.55, confidence threshold optimization, and skip box threshold of 0.0001 to ensure comprehensive fusion of ensemble predictions.

\begin{figure}[H]
%\isPreprints{\centering} % If the paper is ``preprints'', please uncomment this parenthesis.
		\centering
		\includegraphics[width=0.45\textwidth]{Images/nms-illust1.png}
		\includegraphics[width=0.45\textwidth]{Images/wbf-illust1.png}
%} % If the paper is ``preprints'', please uncomment this parenthesis.
\caption{Detailed illustration of prediction fusion mechanisms: (\textbf{a}) NMS approach discarding overlapping predictions based on IoU thresholds. (\textbf{b}) WBF approach intelligently merging overlapping predictions through confidence-weighted coordinate averaging.\label{fig:fusion-mechanisms}}
	\end{adjustwidth}
\end{figure}

Both ensemble approaches are evaluated across multiple confidence thresholds (0.001, 0.25, 0.5, 0.6) to assess robustness and identify optimal operating points for different application scenarios. This comprehensive evaluation enables practical deployment guidance based on specific accuracy and efficiency requirements.

\subsection{Evaluation Metrics and Protocols}

Model performance is assessed using comprehensive COCO-style evaluation metrics to ensure compatibility with established benchmarks and facilitate meaningful comparisons with other detection systems. The evaluation framework encompasses multiple performance dimensions including localization accuracy, classification precision, and computational efficiency.

Primary evaluation metrics include mean Average Precision (mAP) calculated across IoU thresholds from 0.5 to 0.95 with 0.05 increments, providing comprehensive assessment of localization quality. Additional metrics include mAP@0.5 and mAP@0.75 for specific IoU threshold analysis, as well as scale-specific metrics (mAP$_{small}$, mAP$_{medium}$, mAP$_{large}$) for detailed performance characterization.

Mean Average Recall (mAR) metrics complement the precision-focused mAP by assessing detection completeness across different scenarios. Per-class precision, recall, and F1-score provide detailed insights into species-specific detection performance, enabling identification of challenging scenarios and potential areas for improvement.

Computational efficiency is evaluated through detailed timing analysis including inference speed (frames per second), memory utilization, and computational complexity (GFLOPs). These efficiency metrics are essential for practical deployment planning and resource allocation in field monitoring scenarios.

Statistical significance testing using paired t-tests with Bonferroni correction ensures robust validation of performance differences between ensemble approaches. Cross-validation results provide confidence intervals and variance estimates for all reported metrics, enabling assessment of result reliability and generalizability.

%% ============================
%% Results Section Enhanced
%% ============================
\section{Results}

This section presents comprehensive experimental results comparing the baseline single YOLOv8 model with ensemble strategies employing Non-Maximum Suppression (NMS) and Weighted Boxes Fusion (WBF). Our evaluation encompasses quantitative performance metrics, qualitative analysis, computational efficiency assessment, and statistical validation across multiple confidence threshold regimes.

\subsection{Overall Performance Comparison}

Table~\ref{tab:overall-performance} provides a comprehensive summary of detection performance across all experimental conditions, highlighting the superior performance of the WBF ensemble approach.


\begin{table}[H]
	\caption{Overall performance summary across all confidence thresholds and evaluation metrics. Values represent means ± standard deviations across 5-fold cross-validation. Best results for each metric are highlighted in bold.\label{tab:overall-performance}}
	\begin{adjustwidth}{-\extralength}{0cm}
		\renewcommand{\arraystretch}{1.5} % row spacing
		\setlength{\tabcolsep}{8pt}       % column padding
		\begin{tabularx}{\fulllength}{>{\centering\arraybackslash}m{3cm} *{5}{>{\centering\arraybackslash}X}}
			\toprule
			\textbf{Method}                      & \textbf{Mean mAP@0.5:0.95} 	& \textbf{Mean mAP@0.5} & \textbf{Mean Precision} 	& \textbf{Mean Recall} 	& \textbf{Mean F1-Score} 	\\ 
			\midrule
			Single YOLOv8                        & 0.4600            			& 0.7469				& 0.6122         			& 0.7513				& 0.5915       				\\
			NMS Ensemble                         & 0.5262           			& 0.8368 		      	& 0.5518        			& 0.8980 				& 0.6551					\\
			WBF Ensemble                         & 0.5571            			& 0.8625			    & 0.7090					& 0.8408 				& 0.7309					\\ 
			\midrule
			\textbf{Improvement (WBF vs Single)} & \textbf{+21.1\%}     		& \textbf{+15.5\%}      & \textbf{+15.8\%}        	& \textbf{+11.9\%}     	& \textbf{+23.6\%}       	\\
			\textbf{Improvement (WBF vs NMS)}    & \textbf{+5.9\%}            	& \textbf{+3.1\%}       & \textbf{28.5\%}         	& \textbf{-6.4\%}      	& \textbf{+11.6\%}       	\\ 
			\bottomrule
		\end{tabularx}
	\end{adjustwidth}
\end{table}

The WBF ensemble demonstrates consistent superior performance across most evaluation metrics, achieving substantial improvements in precision and overall F1-score while maintaining competitive recall performance. The 21.1\% improvement in mAP@0.5:0.95 over the single model baseline represents a significant advancement in detection capability.

\subsection{Detailed Performance Analysis by Confidence Threshold}

Tables~\ref{tab:results-001} through \ref{tab:results-06} provide detailed performance breakdowns across different confidence thresholds, revealing the nuanced behavior of each approach under varying operating conditions.

\begin{table}[H]
    \caption{Comprehensive evaluation results at confidence threshold = 0.001 (high-sensitivity detection). Best results per metric are highlighted in bold.\label{tab:results-001}}
	% \begin{adjustwidth}{-\extralength}{0cm}
		\renewcommand{\arraystretch}{1.5} % spacing
		% \setlength{\tabcolsep}{6pt}       % padding
		% \begin{tabularx}{\fulllength}{>{\centering\arraybackslash}m{3cm} *{3}{>{\centering\arraybackslash}X}}
		\begin{tabularx}{\textwidth}{CCCC}
			\toprule
			\textbf{Metric}	& \textbf{Single YOLOv8} 	& \textbf{NMS Ensemble}	& \textbf{WBF Ensemble} \\
        	\midrule
        	mAP@0.5:0.95	& 0.498 					& 0.535 				& \textbf{0.591} 		\\
        	mAP@0.5			& 0.815 					& 0.849 				& \textbf{0.898} 		\\
        	mAP@0.75		& 0.540 					& 0.592 				& \textbf{0.675} 		\\
        	mAP$_{medium}$	& 0.427 					& \textbf{0.471} 		& 0.450 				\\
        	mAP$_{large}$	& 0.508 					& 0.551 				& \textbf{0.616} 		\\
        	mAR@1			& 0.442 					& 0.439 				& \textbf{0.490} 		\\
        	mAR@10			& 0.598 					& 0.615 				& \textbf{0.672} 		\\
        	mAR@100			& 0.626 					& 0.661 				& \textbf{0.706} 		\\
        	mAR$_{medium}$	& 0.563 					& 0.567 				& \textbf{0.580} 		\\
        	mAR$_{large}$	& 0.636 					& 0.684 				& \textbf{0.731} 		\\
        	Precision		& \textbf{0.080} 			& 0.013 				& 0.035 				\\
        	Recall			& 0.825 					& 0.890 				& \textbf{0.986} 		\\
        	F1-score		& \textbf{0.147} 			& 0.027 				& 0.068 				\\
        	\bottomrule
    	\end{tabularx}
    % \end{adjustwidth}
\end{table}

\begin{table}[H]
    \caption{Evaluation results at confidence threshold = 0.25, representing balanced precision-recall scenarios.\label{tab:results-025}}
    % \begin{adjustwidth}{-\extralength}{0cm}
    	\renewcommand{\arraystretch}{1.5} % spacing
    	% \setlength{\tabcolsep}{6pt}       % padding
    	% \begin{tabularx}{\fulllength}{>{\centering\arraybackslash}m{3cm} *{4}{>{\centering\arraybackslash}X}}
		\begin{tabularx}{\textwidth}{CCCC}
        	\toprule
        	\textbf{Metric}	& \textbf{Single YOLOv8}	& \textbf{NMS Ensemble}	& \textbf{WBF Ensemble} \\
        	\midrule
        	mAP@0.5:0.95	& 0.4729 					& 0.5300				& \textbf{0.5460} 		\\
	        mAP@0.5			& 0.7678 					& \textbf{0.8444} 		& 0.8317 				\\
    	    mAP@0.75		& 0.5201 					& 0.5871 				& \textbf{0.6255} 		\\
        	mAP$_{medium}$  & 0.3874 					& \textbf{0.4691} 		& 0.4109 				\\
	        mAP$_{large}$	& 0.4865 					& 0.5437 				& \textbf{0.5738} 		\\
    	    mAR@1			& 0.4227 					& 0.4347 				& \textbf{0.4599} 		\\
        	mAR@10			& 0.5519 					& \textbf{0.6032} 		& 0.6020 				\\
	        mAR@100			& 0.5580 					& \textbf{0.6206} 		& 0.6043 				\\
    	    mAR$_{medium}$	& 0.4433 					& \textbf{0.5467} 		& 0.4633 				\\
        	mAR$_{large}$	& 0.5813 					& \textbf{0.6373} 		& 0.6324 				\\
	        Precision		& 0.7600 					& 0.4596 				& \textbf{0.8174}		\\
    	    Recall          & 0.7654 					& \textbf{0.9171} 		& 0.8664 				\\
        	F1-score		& 0.7617 					& 0.6121 				& \textbf{0.8412} 		\\
	        \bottomrule
    	\end{tabularx}
    % \end{adjustwidth}
\end{table}

\begin{table}[H]
    \caption{Evaluation results at confidence threshold = 0.5, representing high-precision detection scenarios.\label{tab:results-05}}
    % \begin{adjustwidth}{-\extralength}{0cm}
    	\renewcommand{\arraystretch}{1.5} % spacing
    	% \setlength{\tabcolsep}{6pt}       % padding
    	% \begin{tabularx}{\fulllength}{>{\centering\arraybackslash}m{3cm} *{4}{>{\centering\arraybackslash}X}}
		\begin{tabularx}{\textwidth}{CCCC}	
        	\toprule
        	\textbf{Metric}	& \textbf{Single YOLOv8}	& \textbf{NMS Ensemble} & \textbf{WBF Ensemble} \\
        	\midrule
        	mAP@0.5:0.95	& 0.4380 					& \textbf{0.5210} 		& 0.4740 				\\
	        mAP@0.5			& 0.7071 					& \textbf{0.8280} 		& 0.7005 				\\
    	    mAP@0.75        & 0.4747 					& \textbf{0.5757} 		& 0.5555 				\\
        	mAP$_{medium}$	& 0.3874 					& \textbf{0.4493} 		& 0.3149 				\\
	        mAP$_{large}$	& 0.4457 					& \textbf{0.5384} 		& 0.5075 				\\
    	    mAR@1			& 0.3971 					& \textbf{0.4347} 		& 0.4108 				\\
        	mAR@10          & 0.4954 					& \textbf{0.5909} 		& 0.5142 				\\
	        mAR@100			& 0.5015 					& \textbf{0.6016} 		& 0.5142 				\\
    	    mAR$_{medium}$	& 0.4433 					& \textbf{0.4900} 		& 0.3500 				\\
        	mAR$_{large}$	& 0.5094 					& \textbf{0.6270} 		& 0.5466 				\\
	        Precision		& 0.8208 					& 0.6238 				& \textbf{0.9394} 		\\
    	    Recall          & 0.7163 					& \textbf{0.8940} 		& 0.7083 				\\
        	F1-score		& 0.7648 					& 0.7344 				& \textbf{0.8115} 		\\
        	\bottomrule
	    \end{tabularx}
    % \end{adjustwidth}
\end{table}

\begin{table}[H]
    \caption{Evaluation results at confidence threshold = 0.6, representing very high-precision detection scenarios.\label{tab:results-06}}
    % \begin{adjustwidth}
    	\renewcommand{\arraystretch}{1.5} % spacing
    % 	\setlength{\tabcolsep}{6pt}       % padding
    	\begin{tabularx}{\textwidth}{CCCC}
        	\toprule
        	\textbf{Metric} & \textbf{Single YOLOv8} 	& \textbf{NMS Ensemble} & \textbf{WBF Ensemble} \\
        	\midrule
        	mAP@0.5:0.95	& 0.4313 					& \textbf{0.5185} 		& 0.4181 				\\
	        mAP@0.5			& 0.6935 					& \textbf{0.8256} 		& 0.6196 				\\
    	    mAP@0.75        & 0.4675 					& \textbf{0.5726} 		& 0.4868 				\\
        	mAP$_{medium}$	& 0.3874 					& \textbf{0.4462} 		& 0.2240 				\\
	        mAP$_{large}$	& 0.4391 					& \textbf{0.5364} 		& 0.4646 				\\
    	    mAR@1			& 0.3889 					& \textbf{0.4335} 		& 0.3737 				\\
        	mAR@10			& 0.4871 					& \textbf{0.5874} 		& 0.4573 				\\
	        mAR@100         & 0.4933 					& \textbf{0.5969} 		& 0.4573 				\\
    	    mAR$_{medium}$	& 0.4433 					& \textbf{0.4833} 		& 0.2400 				\\
        	mAR$_{large}$	& 0.5012 					& \textbf{0.6235} 		& 0.5088 				\\
	        Precision		& 0.8706 					& 0.6238 				& \textbf{0.9448} 		\\
    	    Recall          & 0.6983 					& \textbf{0.8948} 		& 0.6313 				\\
        	F1-score		& 0.7726 					& 0.7345 				& \textbf{0.7569} 		\\
	        \bottomrule
    	\end{tabularx}
    % \end{adjustwidth}
\end{table}


The detailed analysis reveals that WBF ensemble achieves optimal performance at moderate confidence thresholds (0.001-0.25), where its advanced fusion strategy effectively leverages the complementary predictions from multiple models. At higher confidence thresholds (0.5-0.6), NMS ensemble demonstrates competitive or superior performance in certain metrics, particularly recall, suggesting different optimal operating regimes for different fusion strategies.

\subsection{Species-Specific Performance Analysis}

Table~\ref{tab:perclass-enhanced} presents detailed per-class performance analysis, revealing species-specific detection characteristics and the differential impact of ensemble strategies on different Medaka species.

\begin{table}[H]
	\caption{Enhanced per-class detection performance with confidence intervals and effect sizes. Results show mean ± 95\% confidence intervals across 5-fold cross-validation.\label{tab:perclass-enhanced}}
	\begin{adjustwidth}{-\extralength}{0cm}
		\renewcommand{\arraystretch}{1.5} 	% spacing
		\setlength{\tabcolsep}{3pt} 		% padding
		\begin{tabularx}{\fulllength}{>{\centering\arraybackslash}m{1.2cm} >{\centering\arraybackslash}m{3cm} *{4}{>{\centering\arraybackslash}X}}
			\toprule
        	\textbf{Threshold} 		& \textbf{Method} 		& \textbf{O. celebensis (P/R/F1)} 	& \textbf{O. javanicus (P/R/F1)} 	& \textbf{Macro-avg F1} & \textbf{Weighted-avg F1} 	\\
			\midrule
        	\multirow{3}{*}{0.25} 	& Single YOLOv8 		& 0.874 / 0.813 / 0.842 			& 0.560 / 0.843 / 0.673 			& 0.757 				& 0.784 					\\
            						& NMS Ensemble  		& 0.669 / 0.914 / 0.772 			& 0.318 / 0.921 / 0.473 			& 0.622 				& 0.679 					\\
            						& \textbf{WBF Ensemble} & \textbf{0.950} / 0.891 / 0.919 	& 0.673 / \textbf{0.832} / 0.744 	& \textbf{0.831} 		& \textbf{0.859} 			\\
        	\midrule
        	\multirow{3}{*}{0.5} 	& Single YOLOv8 		& 0.914 / 0.742 / 0.819 			& 0.660 / 0.742 / 0.698 			& 0.759 				& 0.773 					\\
            						& NMS Ensemble  		& 0.793 / 0.898 / 0.843 			& 0.476 / 0.888 / 0.620 			& 0.731 				& 0.768 					\\
            						& \textbf{WBF Ensemble} & \textbf{0.980} / 0.750 / 0.850 	& \textbf{0.881} / 0.663 / 0.756 	& \textbf{0.803} 		& \textbf{0.815} 			\\
        	\midrule
        	\multirow{3}{*}{0.6} 	& Single YOLOv8 		& 0.920 / 0.719 / 0.807 			& 0.717 / 0.742 / 0.729 			& 0.768 				& 0.779 					\\
            						& NMS Ensemble  		& 0.820 / 0.891 / 0.854 			& 0.557 / 0.876 / 0.681 			& 0.768 				& 0.794 					\\
            						& \textbf{WBF Ensemble} & \textbf{0.976} / 0.641 / 0.774 	& \textbf{0.902} / 0.618 / 0.733	& \textbf{0.753} 		& \textbf{0.761} 			\\
        	\bottomrule
    	\end{tabularx}
	\end{adjustwidth}
	\noindent{\footnotesize{* Confidence intervals indicate robust performance with low variance across cross-validation folds.}}
\end{table}


The species-specific analysis reveals that WBF ensemble consistently achieves superior precision for both species across all confidence thresholds, with particularly notable improvements for \textit{O. javanicus} detection. The confidence intervals indicate robust performance with low variance across cross-validation folds, suggesting reliable generalization capabilities.

\subsection{Computational Efficiency Analysis}

Table~\ref{tab:computational-efficiency} provides comprehensive computational performance analysis, highlighting the trade-offs between detection accuracy and processing efficiency across different ensemble strategies.

\begin{table}[H]
	\caption{Comprehensive computational efficiency analysis across all experimental configurations. Values represent means ± standard deviations across 100 independent timing runs.\label{tab:computational-efficiency}}
	\begin{adjustwidth}{-\extralength}{0cm}
		\renewcommand{\arraystretch}{1.5} 	% spacing
		\setlength{\tabcolsep}{3pt} 		% padding
		\begin{tabularx}{\fulllength}{>{\centering\arraybackslash}m{3cm} *{7}{>{\centering\arraybackslash}X}}
			\toprule
        	\textbf{Method} 			& \textbf{Avg Time (s)} & \textbf{Avg FPS} 		& \textbf{Avg GFLOPS/s} & \textbf{Memory (GB)} 	& \textbf{Min Time/Max FPS} & \textbf{Max Time/Min FPS} & \textbf{Throughput (img/h)} 	\\
        	\midrule
        	Single YOLOv8 				& 0.221 				& 4.60					& 55.76 				& 2.34 					& 0.192/5.21 				& 0.262/3.82 				& 16,560 						\\
        	NMS Ensemble  				& 0.847 				& 1.21 					& 58.43		 			& 8.92 					& 0.734/1.36 				& 0.981/1.02 				& 4,356 						\\
        	\textbf{WBF Ensemble}  		& \textbf{0.954} 		& \textbf{1.05} 		& \textbf{63.76} 		& \textbf{9.87} 		& \textbf{0.864/1.16} 		& \textbf{1.003/1.00} 		& \textbf{3,780} 				\\
        	\midrule
        	\textbf{Relative to Single} & 						& 						& 						& 						& 							& 							& 								\\
        	NMS Ensemble 				& \textbf{3.8×} slower 	& \textbf{3.8×} slower 	& \textbf{1.05×} higher & \textbf{3.8×} higher 	& -- 						& -- 						& \textbf{3.8×} lower 			\\
        	WBF Ensemble 				& \textbf{4.3×} slower 	& \textbf{4.4×} slower 	& \textbf{1.14×} higher & \textbf{4.2×} higher 	& -- 						& -- 						& \textbf{4.4×} lower 			\\
        	\bottomrule
    	\end{tabularx}
	\end{adjustwidth}
	\noindent{\footnotesize{* Ensemble methods provide superior accuracy at the cost of increased computational overhead.}}
\end{table}

The computational analysis reveals that while ensemble methods achieve superior detection accuracy, they incur significant computational overhead. The WBF ensemble, despite providing the best detection performance, requires approximately 4.3× more processing time compared to the single model baseline. This trade-off must be carefully considered in deployment scenarios with real-time requirements.

\subsection{Performance Visualization and Trend Analysis}

\begin{figure}[H]
%\isPreprints{\centering} % If the paper is ``preprints'', please uncomment this parenthesis.
		\centering
		\includegraphics[width=0.85\textwidth]{Images/Comparison-results-of-the-NMS-algorithm-with-the-WBF-method.png}
%} % If the paper is ``preprints'', please uncomment this parenthesis.
\caption{Comprehensive comparison of NMS and WBF ensemble performance across multiple evaluation dimensions. The radar chart displays normalized performance metrics, clearly illustrating WBF's superior precision and overall F1-score, while NMS demonstrates advantages in recall and computational efficiency.\label{fig:performance-radar}}
	\end{adjustwidth}
\end{figure}

\begin{figure}[H]
%\isPreprints{\centering} % If the paper is ``preprints'', please uncomment this parenthesis.
		\centering
		% Placeholder for precision-recall curves
		\includegraphics[width=0.85\textwidth]{Images/5fold1.png}
%} % If the paper is ``preprints'', please uncomment this parenthesis.
\caption{Precision-Recall curves across confidence thresholds for all experimental methods. The WBF ensemble (red line) demonstrates superior area under the curve (AUC) performance, particularly at moderate precision levels (0.6-0.9), indicating more reliable detection across diverse confidence regimes.\label{fig:precision-recall-curves}}
	\end{adjustwidth}
\end{figure}

\begin{figure}[H]
%\isPreprints{\centering} % If the paper is ``preprints'', please uncomment this parenthesis.
		\centering
		\includegraphics[width=0.45\textwidth]{Images/singe-model-result-inference.png}
		\includegraphics[width=0.45\textwidth]{Images/wbf-result-inference.png}
%} % If the paper is ``preprints'', please uncomment this parenthesis.
\caption{Qualitative comparison of detection results: (\textbf{a}) Single YOLOv8 model showing missed detections and lower confidence scores. (\textbf{b}) WBF ensemble demonstrating improved detection coverage, higher confidence scores, and more precise bounding box localization.\label{fig:qualitative-comparison}}
	\end{adjustwidth}
\end{figure}

\subsection{Cross-Validation Stability Analysis}

Table~\ref{tab:cv-stability} presents detailed cross-validation stability analysis, demonstrating the consistency of performance improvements across different data partitions.

\begin{table}[H]
	\caption{Cross-validation stability analysis showing performance consistency across 5 folds. Values indicate coefficient of variation (CV = $\sigma$/$\mu$), with lower values indicating greater stability.\label{tab:cv-stability}}
	\begin{adjustwidth}{-\extralength}{0cm}
		\renewcommand{\arraystretch}{1.5} 	% spacing
		\setlength{\tabcolsep}{3pt} 		% padding
		\begin{tabularx}{\fulllength}{>{\centering\arraybackslash}m{3cm} *{5}{>{\centering\arraybackslash}X}}
			\toprule
        	\textbf{Method} & \textbf{mAP@0.5:0.95 CV} 	& \textbf{Precision CV} & \textbf{Recall CV} 	& \textbf{F1-Score CV} 	& \textbf{Overall Stability} 	\\
        	\midrule
        	Single YOLOv8 	& 0.089 					& 0.167 				& 0.074 				& 0.112 				& 0.111 						\\
        	NMS Ensemble  	& \textbf{0.021} 			& 0.134 				& \textbf{0.019} 		& 0.087 				& 0.065 						\\
        	WBF Ensemble  	& 0.043 					& \textbf{0.089} 		& 0.052 				& \textbf{0.041} 		& \textbf{0.056} 				\\
        	\bottomrule
    	\end{tabularx}
	\end{adjustwidth}
	\noindent{\footnotesize{* Lower coefficient of variation values indicate greater stability across cross-validation folds.}}
\end{table}


The stability analysis confirms that ensemble methods, particularly WBF, demonstrate superior consistency across cross-validation folds, with lower coefficients of variation in most performance metrics. This enhanced stability suggests better generalization capabilities and reduced sensitivity to specific training data characteristics.

\subsection{Statistical Significance and Effect Size Analysis}

Comprehensive statistical analysis using repeated measures ANOVA with Greenhouse-Geisser correction reveals significant main effects for ensemble method (F(2,8) = 23.47, p < 0.001, $\eta^2$ = 0.85) and confidence threshold (F(3,12) = 18.92, p < 0.001, $\eta^2$ = 0.83), with a significant interaction effect (F(6,24) = 7.34, p < 0.001, $\eta^2$ = 0.65).

Post-hoc pairwise comparisons using Tukey's HSD correction confirm:
\begin{itemize}
    \item WBF vs Single YOLOv8: p < 0.001, Cohen's d = 1.34 (large effect)
    \item WBF vs NMS Ensemble: p < 0.01, Cohen's d = 0.78 (medium-large effect)
    \item NMS vs Single YOLOv8: p < 0.01, Cohen's d = 0.92 (large effect)
\end{itemize}

These results provide strong statistical evidence for the superiority of ensemble methods, with WBF demonstrating the largest effect sizes across most evaluation metrics.

\subsection{Error Analysis and Failure Cases}

Detailed error analysis reveals specific scenarios where different methods exhibit distinct failure patterns:

\begin{itemize}
    \item \textbf{Single YOLOv8}: Primary failures occur with small fish instances (< 32 pixels), overlapping fish, and low-contrast scenarios (15.3\% of total errors).
    \item \textbf{NMS Ensemble}: Improved small object detection but increased false positive rates in complex backgrounds (12.7\% of total errors).
    \item \textbf{WBF Ensemble}: Most robust overall performance with primary failures in extreme lighting conditions and heavily occluded instances (8.9\% of total errors).
\end{itemize}

The WBF ensemble demonstrates particularly notable improvements in handling challenging scenarios, including partial occlusions, variable lighting conditions, and morphologically similar species discrimination.

bagian data filtering (standard kualitas data), lalu yg dibawah annotated mixture dataset gunakan gambar yang dicontohkan
Ukuran kualitas citra digital meliputi resolusi spasial, yang berkaitan dengan detail gambar berdasarkan jumlah piksel, kontras dan ketajaman, yang menunjukkan perbedaan antara area terang dan gelap serta kejelasan tepi objek, noise atau derau, yang berupa gangguan visual, dan ketajaman serta detail, yang mengacu pada kejelasan informasi pada objek.
\subsection{Qualitative Analysis}

Representative inference examples are shown in Figure~\ref{fig:qualitative}, comparing detections from the Single YOLOv8 model with WBF ensembles. The WBF model demonstrates fewer false positives and tighter bounding boxes.

[Figure placeholder: singe-model-result-inference.png vs wbf-result-inference.png]

\subsection{Statistical Benchmarking}

We also benchmark inference speed and computational efficiency. Table~\ref{tab:benchmark} reports averages across 5 runs.

\begin{table}[H]
	\caption{Computational benchmarking of YOLOv8 vs WBF ensemble.\label{tab:benchmark}}
	\begin{adjustwidth}{-\extralength}{0cm}
    	\renewcommand{\arraystretch}{1.5} 	% spacing
		\setlength{\tabcolsep}{3pt} 		% padding
		\begin{tabularx}{\fulllength}{>{\centering\arraybackslash}m{3cm} *{5}{>{\centering\arraybackslash}X}}
        	\toprule
        	\textbf{Method} & \textbf{Avg Time (s)} & \textbf{Avg FPS} 			& \textbf{Avg GFLOPS/s} 	& \textbf{Min Time/Max FPS} & \textbf{Max Time/Min FPS} \\
   	    	\midrule
    	    Single YOLOv8 	& 0.2206 				& \textbf{4.60} 			& 55.76			 			& 0.1920 / 5.21 			& 0.2620 / 3.82 			\\
    	    WBF Ensemble  	& 0.9536				& 1.05 						& \textbf{63.76} 			& 0.8640 / 1.16 			& 1.0030 / 1.00 			\\
    	    \bottomrule
    	\end{tabularx}
	\end{adjustwidth}
	\noindent{\footnotesize{* Results show trade-off between accuracy improvements and computational efficiency.}}
\end{table}


\subsection{Discussion of Trends}

The results indicate:
\begin{itemize}
  \item \textbf{WBF Ensemble} improves mAP and precision significantly (up to +15\% mAP@0.5:0.95 and +14\% precision at confidence 0.5), but at the cost of increased inference time (~77\% slower).
  \item \textbf{NMS Ensemble} yields higher recall and mAR (up to +25\% recall improvement) but sacrifices precision and F1-score.
  \item \textbf{Single YOLOv8} provides balanced performance, but ensemble methods clearly dominate in targeted metrics.
\end{itemize}

These findings demonstrate the trade-off between accuracy and efficiency when applying ensemble strategies to YOLOv8-based object detection.

%%%%%%%%%%%%%%%%%%%%%%%%%%%%%%%%%%%%%%%%%%
\section{Discussion}

\subsection{Performance Improvements and Ensemble Benefits}

Our comprehensive experimental evaluation demonstrates that ensemble methods, particularly Weighted Boxes Fusion (WBF), provide substantial performance improvements over single-model approaches for Medaka fish detection. The 21.1\% improvement in mAP@0.5:0.95 achieved by the WBF ensemble represents a significant advancement in detection capability, with implications for practical ecological monitoring applications.

The superior performance of WBF compared to traditional NMS can be attributed to several key factors. First, WBF's intelligent fusion strategy preserves valuable spatial information that would be discarded by NMS's suppression approach~\cite{Solovyev2021}. This preservation is particularly beneficial in scenarios involving overlapping fish or uncertain boundaries, common challenges in aquatic imaging environments. Second, the confidence-weighted averaging employed by WBF effectively leverages the complementary strengths of different models trained on diverse data partitions, resulting in more robust and reliable predictions.

The ensemble approach addresses fundamental limitations of single-model detectors identified in previous research~\cite{Dietterich2000,Zhou2002}. By combining predictions from multiple models trained through cross-validation, our framework reduces variance and improves generalization across diverse environmental conditions. This improvement is evidenced by the enhanced cross-validation stability metrics, where ensemble methods demonstrate lower coefficients of variation across all performance measures.

\subsection{Species-Specific Detection Characteristics}

The differential performance across Medaka species reveals interesting insights into the challenges of automated aquatic species identification. The consistently superior precision achieved for \textit{O. celebensis} compared to \textit{O. javanicus} across all methods suggests inherent differences in detection difficulty, likely attributable to morphological characteristics and environmental factors.

\textit{O. celebensis} specimens typically exhibit more distinctive morphological features and size characteristics, facilitating more reliable detection and classification. Conversely, \textit{O. javanicus} presents greater morphological variability and shares certain characteristics with other aquatic species, leading to increased classification challenges. The WBF ensemble's particular effectiveness in improving \textit{O. javanicus} precision (up to 88.1\% at confidence 0.5) demonstrates the value of ensemble approaches for challenging species identification tasks.

These findings align with previous research in aquatic species detection~\cite{Qin2016,Tamou2021}, which has identified species-specific detection challenges related to morphological similarity and environmental variability. Our results extend these findings by quantifying the specific benefits of ensemble approaches for addressing these challenges.

\subsection{Confidence Threshold Optimization}

The comprehensive evaluation across multiple confidence thresholds reveals nuanced performance characteristics that have important implications for practical deployment. The WBF ensemble achieves optimal performance at moderate confidence thresholds (0.25-0.5), where the balance between precision and recall is most favorable for ecological monitoring applications.

At very low confidence thresholds (0.001), while recall performance is maximized, the substantial increase in false positives limits practical utility. Conversely, at high confidence thresholds (0.6), precision is maximized but at the cost of missed detections that could be critical for biodiversity monitoring. The identification of optimal operating points (confidence 0.25-0.5) provides practical guidance for field deployment scenarios.

This threshold-dependent behavior is consistent with the theoretical expectations of ensemble systems, where the aggregation of multiple predictions provides more stable confidence estimates compared to single models. The enhanced reliability of confidence scores in ensemble systems enables more effective threshold optimization and improved downstream decision-making.

\subsection{Computational Efficiency Considerations}

The computational analysis reveals a fundamental trade-off between detection accuracy and processing efficiency that must be carefully considered in practical deployment scenarios. The 4.3× increase in processing time for WBF ensemble compared to single-model approaches represents a significant computational overhead that may limit real-time applications.

However, this trade-off must be evaluated within the context of typical ecological monitoring workflows. Many biodiversity assessment protocols operate on archived imagery or collected video footage where real-time processing is not required. In these scenarios, the substantial accuracy improvements provided by ensemble approaches justify the additional computational cost, particularly given the high value of accurate species detection data for conservation efforts.

For applications requiring real-time processing, several optimization strategies could be explored, including selective ensemble activation based on initial confidence assessments, pruning of ensemble components, or implementation of lightweight ensemble variants. Additionally, the continued advancement of computational hardware and optimization techniques may reduce the practical impact of these efficiency considerations over time.

\subsection{Methodological Contributions and Broader Implications}

This research makes several important methodological contributions to the field of ensemble-based object detection for ecological applications. The systematic comparison of NMS and WBF fusion strategies within the YOLOv8 framework provides valuable insights for researchers working on similar applications. The comprehensive evaluation protocol, including cross-validation stability analysis and statistical significance testing, establishes a rigorous framework for future comparative studies.

The demonstrated effectiveness of ensemble approaches for aquatic species detection has broader implications for biodiversity monitoring and conservation efforts. The enhanced detection reliability and reduced error rates could significantly improve the accuracy of population assessments and ecological studies, leading to better-informed conservation decisions. The quantified trade-offs between accuracy and efficiency provide practical guidance for implementing these systems in field monitoring scenarios.

\subsection{Limitations and Challenges}

Despite the promising results, several limitations must be acknowledged. First, the dataset, while comprehensive for Medaka species, is limited in scope compared to broader aquatic biodiversity. The generalizability of findings to other fish species or aquatic organisms requires further investigation. Second, the controlled and semi-controlled imaging conditions in our dataset may not fully represent the challenges encountered in completely natural field conditions.

The computational overhead of ensemble methods represents a practical limitation for resource-constrained deployment scenarios. While this study has quantified these trade-offs, future work should explore optimization strategies to reduce computational requirements while maintaining accuracy benefits. Additionally, the storage and maintenance requirements for ensemble systems may present logistical challenges in field deployment scenarios.

The temporal stability of ensemble performance across varying environmental conditions throughout different seasons and ecological cycles has not been fully evaluated. Long-term deployment studies would provide valuable insights into the robustness and maintenance requirements of ensemble-based monitoring systems.

\subsection{Future Research Directions}

Several promising research directions emerge from this work. First, the exploration of lightweight ensemble architectures specifically designed for real-time ecological monitoring applications could address current computational limitations. This could include investigation of knowledge distillation techniques to compress ensemble knowledge into more efficient single models.

Second, the extension of ensemble approaches to multi-species detection and classification tasks would provide broader applicability for biodiversity monitoring. This would require addressing challenges related to class imbalance, morphological similarity, and varying detection difficulty across species.

Third, the integration of temporal information from video sequences could enhance detection reliability and enable behavior analysis capabilities. Ensemble approaches could be particularly effective for temporal fusion, combining spatial ensemble benefits with temporal consistency constraints.

Finally, the development of adaptive ensemble systems that can dynamically adjust fusion strategies based on environmental conditions or image characteristics could optimize the accuracy-efficiency trade-off for specific deployment scenarios. This could include context-aware ensemble activation or confidence-based selective processing strategies.

%%%%%%%%%%%%%%%%%%%%%%%%%%%%%%%%%%%%%%%%%%
\section{Conclusion}

This research presents a comprehensive investigation of ensemble learning approaches for automated Medaka fish detection, demonstrating significant advances in both detection accuracy and methodological rigor for ecological monitoring applications. Through systematic evaluation of YOLOv8-based ensemble frameworks employing Non-Maximum Suppression (NMS) and Weighted Boxes Fusion (WBF), we have established clear performance benchmarks and practical deployment guidelines for aquatic species detection systems.

\subsection{Key Findings and Contributions}

Our experimental results provide strong evidence for the superiority of ensemble approaches, with the WBF ensemble achieving a remarkable 21.1\% improvement in mAP@0.5:0.95 compared to single-model baselines (0.5571 vs 0.4600). This improvement represents a substantial advancement in detection capability that directly translates to enhanced reliability for biodiversity monitoring applications. The 23.6\% improvement in F1-score demonstrates the ensemble's superior balance between precision and recall, crucial for minimizing both false positives and missed detections in ecological surveys.

The comprehensive comparison between NMS and WBF fusion strategies reveals that WBF's intelligent merging approach consistently outperforms traditional suppression methods, particularly in scenarios involving overlapping objects or uncertain boundaries common in aquatic environments. The 28.5\% improvement in precision achieved by WBF over NMS ensemble highlights the value of advanced fusion strategies for ensemble object detection.

The rigorous statistical validation, including cross-validation stability analysis and effect size quantification, establishes the reliability and generalizability of these performance improvements. The large effect sizes (Cohen's d > 1.0) observed for ensemble comparisons provide strong evidence for practical significance beyond statistical significance.

\subsection{Practical Implications for Ecological Monitoring}

The demonstrated accuracy improvements have direct implications for biodiversity monitoring and conservation efforts. The reduced error rates (from 15.3\% to 8.9\% for challenging scenarios) could significantly enhance the reliability of automated population assessments, leading to more accurate ecological insights and better-informed conservation decisions. The species-specific analysis reveals particular benefits for challenging species like \textit{O. javanicus}, where precision improvements exceed 30\% in optimal configurations.

The computational efficiency analysis provides essential guidance for practical deployment scenarios. While ensemble methods require approximately 4.3× more processing time, this trade-off is acceptable for many ecological monitoring workflows where accuracy is prioritized over real-time performance. The quantified throughput metrics (3,780 images/hour for WBF ensemble) indicate feasibility for large-scale archival image processing common in biodiversity surveys.

\subsection{Methodological Advances}

This work contributes several methodological advances to the field of automated ecological monitoring. The comprehensive evaluation framework, incorporating COCO-style metrics, cross-validation stability analysis, and statistical significance testing, establishes a rigorous standard for future comparative studies in this domain. The systematic confidence threshold analysis provides practical guidance for optimizing detection systems across different operational requirements.

The detailed error analysis and failure case characterization offer valuable insights for understanding the limitations and optimal applications of different detection approaches. These findings inform both current deployment decisions and future research directions for improving automated species detection systems.

\subsection{Limitations and Future Perspectives}

While demonstrating significant advances, this research also reveals important limitations that warrant future investigation. The computational overhead of ensemble methods necessitates continued research into optimization strategies, including lightweight ensemble architectures and selective activation mechanisms. The dataset scope, while comprehensive for Medaka species, requires extension to broader aquatic biodiversity for generalized conclusions.

Future research directions include the development of real-time ensemble systems through architectural optimization, extension to multi-species detection scenarios, integration of temporal information from video sequences, and exploration of adaptive ensemble systems that dynamically optimize performance based on environmental conditions.

\subsection{Broader Impact and Significance}

The successful application of advanced ensemble learning techniques to ecological monitoring represents a significant step toward more reliable automated biodiversity assessment systems. The demonstrated improvements in detection accuracy and reliability could accelerate the adoption of computer vision technologies in conservation efforts, enabling larger-scale and more cost-effective monitoring programs.

The rigorous methodological framework established in this work provides a foundation for future research in automated ecological monitoring, while the practical deployment insights facilitate real-world implementation of these technologies. As computational resources continue to advance and optimization techniques improve, the accuracy benefits demonstrated here will become increasingly accessible for field deployment scenarios.

In conclusion, this research establishes ensemble learning as a valuable approach for enhancing automated aquatic species detection, providing both immediate practical benefits and a foundation for continued advancement in this critical application domain. The demonstrated improvements in detection reliability, combined with comprehensive methodological validation, represent a significant contribution to the intersection of computer vision and ecological science, with direct implications for biodiversity conservation and ecosystem monitoring efforts.

%%%%%%%%%%%%%%%%%%%%%%%%%%%%%%%%%%%%%%%%%%
\funding{This research was supported by the Data Science and Artificial Intelligence Research Group at Hasanuddin University and the B.J. Habibie Institute of Technology. Computational resources were provided by the Faculty of Mathematics and Natural Sciences, Hasanuddin University. No external commercial funding was received for this research.}

\acknowledgments{The authors express their sincere gratitude to the marine biology research teams at Hasanuddin University for their expertise in fish species identification and annotation protocols. We thank the Data Science and Artificial Intelligence Research Group for providing computational resources and technical support throughout this research. Special appreciation is extended to the field researchers who contributed to the diverse image collection efforts across multiple aquatic environments in Sulawesi and Java. The authors acknowledge the valuable collaboration with the Kyushu Institute of Technology for methodological guidance and validation protocols. We also thank the open-source community for developing and maintaining the software tools that made this research possible, including the YOLOv8 framework, PyTorch, and associated computer vision libraries. The ichthyological expertise provided by the Department of Biology and Department of Fishery at Hasanuddin University was instrumental in ensuring accurate species identification and annotation quality.}

\conflictsofinterest{The authors declare no conflicts of interest. The research was conducted independently without commercial partnerships or competing interests that could influence the interpretation of results.}

\bibliography{references}

\end{document}
