\subsection{AdaBoost Ensemble Method}
One of the main contributions of this research is the application of the AdaBoost ensemble method to the YOLOv8 model. The AdaBoost algorithm mitigates the limitations of individual detectors by assigning higher weights to misclassified samples in subsequent iterations. This iterative reweighting proved particularly beneficial for challenging Medaka images, such as those with small body sizes, partial occlusions, or variable lighting conditions.  

Compared with previous studies that relied on single-model detectors for fish recognition, ensemble learning has been shown to consistently improve robustness in object detection tasks \cite{refX,refY}. Our findings align with these trends, showing that the YOLOv8–AdaBoost ensemble achieved higher mAP and recall values than any individual model, suggesting that boosting strategies can play a critical role in addressing class imbalance and small-object detection in aquatic environments.  

\subsection{5-Fold Cross-Validation for Model Generalization}
Another major contribution of this study is the implementation of 5-fold cross-validation, which ensured robust and unbiased model evaluation. By rotating validation across five subsets, we minimized the risk of overfitting to specific data partitions. The stable performance across folds highlights the model’s ability to generalize well, even under variations in background complexity, fish pose, and image quality.  

In line with prior works in ecological computer vision, cross-validation has been recognized as a critical technique to validate models when datasets are relatively small or heterogeneous \cite{refZ}. Our results extend this evidence to Medaka fish detection, confirming that reliable generalization is achievable despite the dataset’s modest size.

\subsection{Implications for Conservation and Monitoring}
The integration of AdaBoost and cross-validation into a YOLOv8-based system for endangered species detection has important implications for conservation monitoring. Accurate identification of small, visually similar fish species enables more reliable assessments of population dynamics, habitat quality, and biodiversity. This is particularly relevant for the genus \textit{Oryzias}, which includes species of conservation concern in Southeast Asia.  

By reducing missed detections of rare or occluded individuals, our approach supports ecological monitoring protocols where reliability is prioritized over speed. The ensemble system is well suited for offline analysis of field-collected data, complementing existing conservation strategies such as population surveys, habitat mapping, and long-term biodiversity monitoring.

\subsection{Future Research Directions}
While the ensemble method improved accuracy, it introduced higher computational costs, making real-time deployment less feasible. Future work should explore lightweight ensemble techniques, model pruning, or knowledge distillation to balance accuracy and efficiency. Moreover, expanding the dataset with additional Medaka species and environmental conditions would improve the system’s scalability and transferability.  

Another promising avenue is integrating temporal information from videos rather than treating frames independently. Temporal coherence may further reduce false negatives for moving fish. Finally, future studies could compare boosting ensembles with other aggregation strategies, such as bagging or weighted box fusion, to identify optimal solutions for aquatic species detection.

%%%%%%%%%%%%%%%%%%%%%%%%%%%%%%%%%%%%%%%%%%
\section{Conclusions}

This study demonstrated the effectiveness of integrating AdaBoost and 5-fold cross-validation with the YOLOv8 model for detecting and classifying rare Medaka fish species. These enhancements significantly improved precision, recall, and mAP, addressing challenges such as small-object detection and visually complex underwater environments. By strengthening robustness and reducing overfitting, the proposed approach contributes to more reliable monitoring of endangered species, supporting conservation initiatives that rely on accurate ecological data.  

However, the improved accuracy came at the expense of computational efficiency, limiting real-time applicability. Future research should focus on lightweight ensemble strategies, alternative boosting algorithms such as Gradient Boosting or XGBoost, and hybrid deep learning approaches that balance accuracy with speed. Expanding datasets across additional \textit{Oryzias} species and incorporating temporal information from video streams may further enhance model generalization. Ultimately, this work establishes a methodological baseline for developing optimized, conservation-oriented object detection systems in aquatic environments.

%%%%%%%%%%%%%%%%%%%%%%%%%%%%%%%%%%%%%%%%%%
\vspace{6pt} 

%%%%%%%%%%%%%%%%%%%%%%%%%%%%%%%%%%%%%%%%%%
%% optional
%\supplementary{The following supporting information can be downloaded at:  \linksupplementary{s1}, Figure S1: title; Table S1: title; Video S1: title.}

% Only for journal Methods and Protocols:
% If you wish to submit a video article, please do so with any other supplementary material.
% \supplementary{The following supporting information can be downloaded at: \linksupplementary{s1}, Figure S1: title; Table S1: title; Video S1: title. A supporting video article is available at doi: link.}

% Only used for preprtints:
% \supplementary{The following supporting information can be downloaded at the website of this paper posted on \href{https://www.preprints.org/}{Preprints.org}.}

% Only for journal Hardware:
% If you wish to submit a video article, please do so with any other supplementary material.
% \supplementary{The following supporting information can be downloaded at: \linksupplementary{s1}, Figure S1: title; Table S1: title; Video S1: title.\vspace{6pt}\\
%\begin{tabularx}{\textwidth}{lll}
%\toprule
%\textbf{Name} & \textbf{Type} & \textbf{Description} \\
%\midrule
%S1 & Python script (.py) & Script of python source code used in XX \\
%S2 & Text (.txt) & Script of modelling code used to make Figure X \\
%S3 & Text (.txt) & Raw data from experiment X \\
%S4 & Video (.mp4) & Video demonstrating the hardware in use \\
%... & ... & ... \\
%\bottomrule
%\end{tabularx}
%}