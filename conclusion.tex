\section{Conclusion}
In this research, we applied AdaBoost and 5-fold cross-validation techniques to enhance the YOLOv8 model for the detection and classification of rare Medaka fish species. These contributions are particularly significant in the context of species conservation, where the accurate identification of endangered animals is critical. Our experiments demonstrated that combining AdaBoost with YOLOv8 improved overall precision and recall, making the model more robust in complex environments, such as underwater habitats where visual challenges are frequent.

Additionally, the use of 5-fold cross-validation allowed for a more thorough evaluation of the model’s performance, minimizing the risk of overfitting and enhancing the generalization capability. These techniques resulted in more accurate predictions, particularly for small objects, addressing a notable gap in the application of object detection algorithms to conservation challenges.

While boosting algorithms like AdaBoost provide substantial performance improvements, their computational cost should be considered, especially for real-time applications. Future work can explore other boosting techniques, such as XGBoost or Gradient Boosting, to further optimize the model's performance without excessively increasing computational demands. Hybrid approaches that combine ensemble learning with architectural innovations in deep learning could yield even more efficient models for endangered species detection and other complex image classification tasks.

