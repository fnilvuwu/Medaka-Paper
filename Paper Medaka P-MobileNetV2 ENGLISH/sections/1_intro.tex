\textit{Oryzias celebensis} and \textit{Oryzias javanicus} are two endemic fish species that play a crucial role in the aquatic ecosystems of Indonesia, particularly in the Sulawesi and Java regions \cite{parenti2008phylogenetic,herder2022more}. Both species play an important role in maintaining biodiversity and ecosystem balance \cite{nurdin2023perubahan}. Identifying rare fish species is a crucial first step in conservation efforts. By recognizing morphological characteristics, genetic analysis, and behavioral understanding, we can gain better insight into these species. Identification also helps us take more effective action to protect habitats and prevent extinction. 

The genus Oryzias belongs to the family Adrianichthyidae. These fish are widely distributed across South Asia, East Asia, and Southeast Asia. Their natural habitat includes rice fields, ponds, ditches, and lakes \cite{kottelat2013fishes,magtoon2009revised}. \textit{Oryzias Celebensis} and \textit{Oryzias Javanicus} are two species particularly threatened with extinction in Indonesia. The Celebes Medaka (Oryzias celebensis) is an endemic species to Sulawesi Island \cite{parenti2011endemism}. Unfortunately, the Medaka fish is an endemic species threatened with extinction and has been officially declared as such by the International Union for Conservation of Nature (IUCN). Conservation efforts are critically important to prevent the extinction of these species. Oryzias javanicus: This fish is also known as the Java Medaka. \cite{lamba2023habitat}

Unfortunately, we realize that the populations of Oryzias celebensis and Oryzias javanicus are declining, pushing these two species to the brink of extinction. This situation is even more concerning because both are categorized as rare fish species endemic to Indonesia, specifically found in Sulawesi and Java. Threats to their natural habitats, climate change, and human activities that damage aquatic environments are further worsening their condition. \cite{MOKODONGAN2015150}In efforts to protect and conserve these rare species, researchers often face ethical dilemmas during the identification process. Conventional methods commonly used, such as examining dead fish or taking fish out of water—which may harm them—are no longer acceptable in the context of environmental sustainability and conservation \cite{voss2014assessing}.

To address this challenge, innovations in artificial intelligence, particularly deep learning technology, offer a promising solution. By leveraging advanced computational capabilities, researchers can use deep learning models to classify and identify fish species—including Oryzias celebensis and Oryzias javanicus—quickly and efficiently. This approach is not only more environmentally friendly but also enables more effective monitoring of these rare fish populations, supporting conservation efforts to ensure their survival in the future.
