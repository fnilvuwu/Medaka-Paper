Based on Table 2, the P-MobileNetV2 model has better evaluation performance results compared to the other models, with Sensitivity = 98.46, Precision = 98.46, F1 Score = 98.46, and Accuracy = 98.7. In this study, we have also applied a padding technique to our dataset. This technique involves adding artificial border values to cover empty spaces, ensuring that the images remain square (1:1 aspect ratio), which helps preserve spatial information during the convolution process. As a result, the P-MobileNetV2 and P-VGG16 models trained on the padded dataset achieved higher metrics compared to models trained without dataset padding, demonstrating that this technique can effectively improve model performance. This can be observed in Figures 10 and 12, where both graphs show notably better trends than those in Figures 9 and 11.

Furthermore, our research provides insight that the padding technique not only improves model accuracy but also helps maintain consistency and preserves spatial context in the dataset images. This is particularly crucial for tasks such as segmentation and object detection, where spatial information in images or datasets is highly important. Therefore, the implementation of the padding technique in this study has proven to be a critical step in enhancing model performance and achieving strong results in our classification task.
